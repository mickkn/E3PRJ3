\chapter{Konklusion}

Projektets formål har været at udvikle et automatiseret vandingssystem til golfbaner kaldet EasyWater8000. Dette vandingssystem skal vha. sensorer vande når der er brug for det og blokere vandingen når der er folk på de enkelte golfhuller. 

Der er opnået en prototype, som fungerer efter hensigten. Enheden fungerer autonomt og starter vandingen når det er nødvendigt. Aktiveres bevægelsessensoren blokeres vandingen. Gruppen er meget tilfreds med at have en funktionsdygtig  prototype, der opfylder de grundlæggende krav til projektet.  

Gruppen har arbejdet seriøst og fremadrettet med projektet. I forbindelse med brug af ASE-modellen har gruppen været delt op i den fagspecifikke fase. Det har givet muligheden for at de enkelte grupper kunne fordybe sig i bestemte områder, og herved bidrage til det samlede produkt.

Gruppemøderne inklusiv de afholdte trivselsrunder har sørget for at vi som gruppe har haft det godt med hinanden og eventuelle problemer kunne afklares i tide. Optil udmeldelsen af Jeppe Stærk var der en del frustration omkring hvorvidt han havde tænkt sig at genoptage projektarbejdet og skolen generelt. Efter udmeldelsen af Jeppe Stærk kunne gruppen igen arbejde koncentreret og fremadrettet, det betød også at hele opgavefordelingen blev genfordelt.  

Det kan konkluderes af udviklingsværktøjet \LaTeX \ er rigtig smart ved store gruppeopgaver, da man sparer meget tid på opsætningen af rapporten og layout.






