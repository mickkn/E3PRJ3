
\chapter{Konklusion}

Projektets formål har været at udvikle et automatiseret vandingssystem til golfbaner kaldet EasyWater8000. Dette vandingssystem skal vha. sensorer vande når der er brug for det og blokere vanding når der er folk på de enkelte golfhuller. 

Der er opnået en prototype, som fungerer efter hensigten. Enheden fungerer autonomt og starter vandingen når det er nødvendigt. Aktiveres bevægelsessensoren blokeres vandingen.  

Gruppen har arbejdet seriøst og fremadrettet med projektet. I forbindelse med brug af ASE-modellen har gruppen været delt op i den fagspecifikkefase. 

Gruppemøderne inklusiv de afholdte trivsels runder har sørget for at vi som gruppe har haft det godt med hinanden og eventuelle problemer kunne afsluttes i tide. 


