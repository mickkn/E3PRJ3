\chapter{Udviklingsværktøjer (PO)} \label{head:udviklingsvaektoejer}
I det følgende afsnit beskrives de benyttede udviklingsværktøjer i forbindelse med dette projekt. 

\section{LaTeX og Texmaker}
Både denne rapport og tilhørende dokumentation er skrevet i sproget \LaTeX. Størstedelen af gruppen havde erfaring med \LaTeX\ fra sidste semester projekt. 

\LaTeX\ er et tekstbaseret kodesprog som gør brugeren fri af layout således at fokus kan rettes mod indholdet. Det kræver selvfølgelig at man sætter sig ind i dette sprog og overholder kodestandarden. 

Texmaker er benyttet som teksteditor til \LaTeX. Denne editor gør det muligt at navigere rundt imellem forskellige \verb+.tex+-filer og bygge dokumenterne via Texmakers indbyggede pdf-viewer.

\section{Multisim}
National Instruments Multisim er benyttet til kredsløbsdesign og simulering af disse. 

\section{PSoC Creator}
PSoC Creator fra Cypress er benyttet i forbindelse med softwareprogrammeringen af Enheden (PSoC4). PSoC4 er Cypress' nyeste ARM baserede Programmable System-on-Chip med en Cortex-M0 processor. 

\section{Eclipse}
Eclipse er benyttet i forbindelse med softwareprogrammeringen af Masteren (Devkit8000).

\section{Qt Creator}
Qt Creator er benyttet i forbindelse med den grafiske brugerflade på Masteren (Devkit8000).

\section{Visual Studio 13}
Visual Studio 13 fra Microsoft er anvendt til at skrive hardware-uafhængige \verb+C+ klasser til Enheden (PSoC4).

\section{Logic}
Logic fra Salae er en logik analysator der er benyttet til at gemme digitale signaler, så det herefter er muligt at analyse f.eks. dele af SPI-kommunikationen. Logic kan fortolke SPI-protokollen direkte.

\section{EAGLE}
EAGLEs PCB layout editor er benyttet til design af print til fugt- og temperatursensoren.

\section{Visio}
Microsoft Visio er benyttet til at udarbejde UML- og SysML-diagrammer til kravspecifikationen og softwaredesignet.

\section{Maple}
Maple fra Maplesoft er et matematikprogram som er anvendt til matematiske udregninger og grafer.

\section{Electronic Toolbox}
App, af Marcus Roskosch, til smartphones og tablets, til at hjælpe med komponent-valg på integreret hardware m.m.

\section{Filhåndtering}
Der er benyttet to cloud services i forbindelse med filhåndteringen for dette projekt.

\subsection{GitHub}
GitHub er benyttet til versionsstyring af projektdokumenter dvs. alle \verb+.tex+-filer til dokumentation og rapport, softwarekildekode, UML- og SysML-diagrammer, Multisim-kredsløbsdiagrammer. GitHub sørger for at alle altid har nyeste version. 

\subsection{Google Drev}
Google Drev er benyttet i flere forskellige forbindelser. Fælles dokumenter, mødereferater, logbog, tidsplan og fælles dokumenter i forbindelse med rette-runder.