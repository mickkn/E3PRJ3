
\chapter{Udviklingsværktøjer (PO)} \label{head:udviklingsvaektoejer}
I det følgende afsnit beskrives, de benyttede, udviklingsværktøjer i forbindelse med dette projekt. 

\section{LaTeX og TexMaker}
Både denne rapport og tilhørende dokumentation er skrevet i sproget \LaTeX. Største delen af gruppen havde erfaring med \LaTeX fra sidste semester projekt. 

\LaTeX er et tekstbaseret kodesprog som gør brugeren fri af layout således at fokus kan rettes mod indholdet. Det kræver selvfølgelig at man sætter sig ind i dette sprog og overholder kodestandarden. 

TexMaker er benyttet som teksteditor til \LaTeX. Denne editor gør det muligt at navigere rundt imellem forskellige .tex-filer og bygge dokumenterne via TexMakers indbyggede pdf-viewer.

Da hele dokumentationen skulle laves færdig  kunne det konstateres at alle billeder skulle gemmes i \verb+.pdf+-formatet, i stedet for i \verb+.png+-formatet. Det gav bedre billedkvalitet og \LaTeX\ reagerede hurtigere.

\section{Multisim}
National Instruments Multisim er benyttet til kredsløbsdesign og simulering af disse. 

\section{PSoC Creator}
PSoC Creator fra Cypress er benyttet i forbindelse med softwareprogrammeringen af Enheden (PSoC4). PSoC4 er Cypress' nyeste ARM baserede Programmable System-on-Chip med en Cortex-M0 processor. 

\section{Eclipse}
Eclipse er benyttet i forbindelse med softwareprogrammeringen af Masteren (Devkit8000). Eclipse kræver et Linux baseret operativsystem.

\section{Qt Creator}
Qt Creator er benyttet i forbindelse med den grafiske brugerflade på Masteren (Devkit8000).

\section{Microsoft Visual Studio 13}
Er anvendt til at skrive hardware-uafhængige \verb+C+ klasser til Enheden (PSoC4).

\section{Logic}
Logic fra Salae er en logik analysator der er benyttet til at gemme digitale signaler, så det herefter var muligt at analyse f.eks dele af SPI-kommunikationen. Logic har indbygget SPI-protokol.

\section{EAGLE}
EAGLEs PCB layout editor er benyttet til design af print til FT-sensoren.

\section{Microsoft Visio}
Microsoft Visio er benyttet til at udarbejde UML- og SysML-diagrammer. 

\section{Maple}
Matematik software, til matematiske udregninger og grafer.

\section{Electronic Toolbox}
App af Marcus Roskosch til smartphones og tablets, til at hjælpe med komponent valg på integreret hardware m.m. 

\section{Filhåndtering}
Der er benyttet 2 cloud resultatet væreservices i forbindelse med filhåndteringen for dette projekt

\subsection{GitHub}
GitHub er benyttet til versionsstyring af projektdokumenter dvs. Alle .tex-filer til dokumentation og rapport, softwarekildekode, UML- og SysML-diagrammer, Multisim-kredsløbsdiagrammer. GitHub sørger for at alle altid har nyeste version. 

\subsection{Google Drev}
Google Drev er benyttet i flere forskellige forbindelser. Fælles dokumenter, mødereferater, logbog, tidsplan og fælles dokumenter i forbindelse med rette runder.