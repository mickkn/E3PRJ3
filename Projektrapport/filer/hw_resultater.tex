%hw_resultater:

På hardware-siden er det lykkedes at implementere en funktionsdygtig prototype.
I hardwaregruppen er det lykkedes at få SPI-kommunikationen op at køre, samt at få hentet data ud af den analoge FT-sensor, denne data er via et API behandlet til en temperatur eller relativ fugtighed.
Ydermere er der lavet et forsyningsprint, der forsynes fra en 12 VDC / 2 A PC-strømforsyning, der forsyner de nødvendige komponenter med korrekt spænding.

Den oprindelige Alpha2-pumpe, som vi fik sponsoreret af Grundfos, er desværre ikke mulig at anvende til det tiltænkte formål, så der er i stedet lavet en alternativ løsning, med en pumpe til en boremaskine. Denne styres stadig af relæet som det også var tiltænkt med Alpha2-pumpen. 

