%hw_resultater:

Samlet set er hele projektet endt ud i en prototype af et fuldt integreret vandingssystem til en golfbane. 
I hardwaregruppen er det lykkedes at få SPI-kommunikationen op at køre, samt at få hentet data ud af den analoge FT-sensor og få denne data behandlet til en temperatur eller relativ fugtighed med en API.
Ydermere er der lavet et forsyningsprint, der forsynes fra en 12 VDC / 2 A strømforsyning, der forsyner de nødvendige komponenter med korrekt spænding.

Den oprindelige pumpe, som vi fik sponsoreret af Grundfos, er desværre ikke mulig at anvende til det tiltænkte formål, så der er i stedet lavet en alternativ løsning, med en pumpe til en boremaskine. Denne styres, som det også var tiltænkt den originale pumpen, stadig fra relæet. 

