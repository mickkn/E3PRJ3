% Abstract
\chapter*{Abstract}

The following document describes the work and process of the groups 3rd term project at Aarhus School of Engineering. The purpose of the project is to use and evaluate the methods and subjects taught at this terms courses. The report describes how the product came from an idea to a physical product as well as the details of the product and the methods used. The documentation holds all technical details about the product.

The product developed is an automatic watering system that helps controlling the environment around gold courses. From a computer program the user can control the whole system and set the limits from which the autonomous system reacts. The autonomous units are placed along the fairways with a grid of sensors placed in the ground on the fairway collecting environmental data.

Development is managed with the ASE-model, which is a semi-iterative project management process. Further more is it done on two different platforms namely the Devkit8000 for the user interaction component and the PSoC4 from Cypress as the core of the autonomous system along with specialised hardware for the sensors.

The results include a functional user interface and a fully established data-collection system. Though limited by the implementation of the water pump system, which is suppressed due time restrictions.

\afterpage{\null\newpage} % Indholdsfortegnelse skal starte på højre side