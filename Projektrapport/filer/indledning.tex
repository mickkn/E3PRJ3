\chapter{Indledning}

Denne rapport beskriver udviklingen af et system kaldet EasyWater8000. Systemet er tiltænkt som en hjælp til greenkeeperen på en golfbane, hvor opgaven med at holde græsset pænt varetages automatisk. Begrundelsen for systemets berettigelse er at det tager meget tid for en greenkeeper at kontrollere banens stand i form af fugtighed i græsset.

Systemet udvikles som et destribueret system hvor en hovedcomputer kan indstille grænseværdierne for fugtighed og temperatur på nogle autonome enheder som er placeret rundt på golfhullerne. Disse enheder starter selv vandingen hvis deres sensorer registrerer værdier uden for grænserne.

I processen fra idé til produkt har gruppen arbejdes sammen med at få idéen og lavet alle de indledende dokumenter som kravspecifikation og arkitekturbeskrivelse.

Se kapitel \ref{head:ordliste} for en komplet ordliste over relevante forkortelser og termer brugt i denne rapport samt projektdokumentationen.

Rapporten er delt op i kapitler inspireret af ASE-modellen. Først beskrives systemet og kravene her til, så følger beskrivelse af arbejdsprocessen og udviklingsværktøjer.
Herefter kommer tre store dele med systemarkitekturen, designet og implementeringen med detaljeret beskrivelse af de enkelte faser. Til sidst følges der op på projektets gjorte erfaringer og muligheder for fremtidigt arbejde samt konklusioner og litteraturliste.