% Resume

\chapter*{Resume}
Følgende rapport og tilhørende dokumentation beskriver arbejdet med gruppens semesterprojekt for 3. semester på Aarhus School of Engineering. Formålet med projektet er at afprøve de metoder og emner som semesterets fag har introduceret. Rapporten beskriver arbejdsmetoderne der er brugt fra idé til produkt, det egentlige produkt og udviklingen af dette samt de overvejelser og værktøjer som er benyttet. Dokumentationen indeholder en udtømmende teknisk beskrivelse af systemet.

Udgangspunktet for systemet er at kunne styre vanding af goldbaner på en nem og simpel måde. Fra et computerprogram kan en bruger styre systemets dele og fastlægge grænser for hvornår vanding skal starte mv. Autonome enheder placeres rundt på goldbanens huller og fra disse enheder distribueres et netværk af sensorer som overvåger de miljømæssige data.

Udviklingsforløbet er styret efter ASE-modellen, som er en halv-iterativ projektledelsesmetode. Produket er udviklet på to forskellige platforme. Brugerinteraktion sker igennem Devkit8000-platformen og det autonome system anvender PSoC4-boardet fra Cypress samt egen udviklet hardware til sensor-håndteringen. 

Projektet er endt ud i et funktionelt system som kan aflæse data fra sensorerne og reagerer på disse. Selve vand-pumpe-konstruktionen til vanding af banerne er kun lavet som en simpel konceptuel opstilling.