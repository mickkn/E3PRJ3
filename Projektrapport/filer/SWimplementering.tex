\section{Softwareimplementering}
Efter designet kan programmeringen påbegyndes. Den tager udgangspunkt i de beskrevne klasser og klassediagrammer.
Da vi har valgt at lave den grafiske brugerflade med Qt frameworket\footnote{\url{http://qt-project.org}, Qt Project Hosting [2014-12-10]} anvendes deres API til mange standardfunktionaliteter. Det betyder at \verb+std::string+ typen omskrives til \verb+QString+. 
Det betyder også at vi anvender en række grafiske elementer kaldet \textit{widgets} til at opbygge klasserne. På figur \ref{fig:sw_class_static} er vist det endelige statiske klassediagram. Her fremgår alle klasser og deres metoder og attributter.
Se dokumentationen afsnit 7.8 Software for detaljer.

\figur{1}{sw_class_devkit_static}{Statisk klassediagram for Master (Devkit8000)}{fig:sw_class_static}

Implementeringen er udført med programmerne Qt Creator og PSoC Creator. Se mere om disse i kapitel \ref{head:udviklingsvaektoejer}.

%SPI
\subsection{SPI (MK PO)}

Der er taget udgangspunkt i HAL Exercise 7\footnote{Hardware abstraktioner. Exercise 7: LDD with SPI. Øvelse med SPI-kommunikation} da SPI blev udarbejdet. Hele kildekoden for Kernemodulet er blevet genbrugt med kun få rettelser. Dele af driveren er også blevet genbrugt. I designet for SPI, er der lagt op til at der skal kunne sendes en 8 bit \verb+CHAR+ ved hver datatransmission. Derfor har det været nødvendigt at rette driveren til, så den kunne håndtere dette.

\subsubsection*{API}

SPI APIen er en klasse med metoder, skrevet ud fra klassebeskrivelserne i software designet, se Projektdokumentation afsnit 6.2.3 Klassebeskrivelser. Implementeringen kan findes på bilags-CD, se CD/SW/Master/Kildekode/spi\_api.

\subsubsection*{Handler}

Handleren er en switchcase i en interrupt-rutine skrevet simultant med APIen, til at handle på forskellige kommandoer sendt fra Master via APIen. Den tilgår interrupt-rutinen ved en SPI transmission og handler i switchcasen udfra hvilken \verb+CHAR+ der modtages. Implementeringen kan findes på bilags-CD, se CD/SW/Enhed/Kildekode/spi\_handler.