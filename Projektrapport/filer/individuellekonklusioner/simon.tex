%Konklusion SK
\section{Simon Kirchheiner}
3. semesterprojektet har givet et stort indblik i at arbejde med bl.a. PSoC Creator, diverse sensorer og SPI-kommunikation. Det har været et spændende projekt med mange udfordringer. Jeg er på stærkstrømslinjen og har derfor fokuseret på HW delen i projeket. Det indebar at arbejde i en gruppe med Mick og Poul, hvor vi havde fokus på design af SPI, sprinkler-relæet, tilslutningsprintet og PIR-sensoren. Da jeg ikke har HAL undervisning og kendskab til Devkit8000 var jeg kun med i opstartsfasen af SPI-delen, derefter forsatte jeg med at færdigudvikle tilslutningsprintet og lave PIR-delen.

Jeg tog ansvaret som referant til alle gruppe- og vejledermøder, det har givet et godt grundlag for at få et struktureret arbejde, da alle kan gå ind i logbogen/referatet og se hvad vi snakkede om. Undervisningen på 3. semester har givet os de nødvendige redskaber til at løse de problemer vi kunne risikere at møde. Vi mistede desværre et gruppemedlem efter korttid, men jeg synes vi som gruppe har håndteret dette godt.
Vi fik udarbejdet en funktionsdygtig prototype af hele systemet, som der eventuelt kunne arbejdes videre på til et komplet system. Gruppen har fungeret godt med løbende trivselsrunder og godt samarbejde, det kan skyldes at næsten alle kendte hinanden fra 2. semesterprojektet. Alt i alt synes jeg at projektet har været lærerigt og styrket fagligheden til 4. semester. 