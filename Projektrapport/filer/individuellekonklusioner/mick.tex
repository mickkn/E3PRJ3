%Konklusion MI
\section{Mick Kirkegaard}
%Din tekst her:
Så blev 3. semester-projekt afsluttet, og det har ikke været uden bump på vejen. Vi har dog efter min mening fået udarbejdet et fint og fungerende projekt. Jeg har været rigtig meget inde over SPI-kommunikationen og føler virkelig, jeg har fået lært meget om seriel kommunikation. Det har betydet, at jeg mere eller mindre kun har siddet og programmeret til Devkit og PSoC4, og jeg har nok fundet ud af, at det er der, hvor jeg befinder mig bedst. Selvom jeg synes, at det er meget udfordrende, er det også meget tilfredsstillende for mig at få noget eksisterende hardware til at agere, som man har kodet det til.

Gruppen fik et nyt medlem i gruppen i starten af semesteret, og det har haft sine udfordringer mht. \LaTeX\ og vores arbejdsprocessor, som vi mere eller mindre har genbrugt fra sidste semester, og som han derfor skulle sættes ind i. 

Da gruppen mistede Jeppe et godt stykke inde i processen, gjorde det, at gruppen gik op i limningen. Jeg mener faktisk ikke, at vi er kommet helt tilbage efter det endnu - på trods af flere trivselsrunder. Der er efter min mening stadig knas i krogene.

I bagklogskabens lys burde vi have sigtet meget lavere, så vi kunne have et reelt og færdigt produkt i hånden i stedet for en nedskaleret prototype; så et golfbanesystem var måske ikke lige sagen.