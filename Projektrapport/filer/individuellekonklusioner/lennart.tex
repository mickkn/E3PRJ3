%Konklusion LB
\section{Lennart Balle}
Som ny mand i en eksisterende gruppe var det med en smule nervøsitet, at jeg gik ind i denne gruppe, hvilket skulle vise sig at være unødvendigt. Jeg blev taget godt imod og følte mig velkommen fra starten af. Det var tydeligt at mærke at gruppen var meget ambitiøs, men stadig kunne forholde sig realistisk til, hvad der var muligt at nå, på den tid der var til rådighed. Kommunikation i gruppen har fungeret upåklageligt, lige undtaget nogle få gange hvor man fokuserede meget på sin egen opgave. Dette problem blev dog opdaget hurtigt alle gange under en jævnlig trivsels-runde.

Fagligt har alle været godt udfordret og der har været mange problemer undervejs, dog har gruppen været god til at hjælpe hinanden på tværs af kompetencer. I projektet er anvendt \LaTeX\ som tekstbehandlings-værktøj. Dette har været svært at finde ud af i starten, men viste sig senere at være en meget stor fordel, og har sparet gruppen megen tid i sammensætning af projekt- rapport og dokumentation. I projektgruppen havde jeg, sammen med Jakob Schmidt, ansvaret for fugt- og temperatursensoren, hvor jeg stod mest for hardwaren og Jakob mest for programmering af PSoC4. 

Alt i alt er jeg godt tilfreds med resultatet af dette semesterprojekt, og jeg føler der har været godt styr på det hele vejen igennem.