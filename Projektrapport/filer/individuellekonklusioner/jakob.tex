%Konklusion JS
\section{Jakob Schmidt}

3. semesterprojekt har for mig været udfordrende fagligt, specielt den serielle kommunikation har været en udfordring. I starten var der valgt en digital sensor, til datamåling af temperatur og fugtighed, men for at få en analog komponent med i projektet faldt valget på SHT21p. Dette gjorde implementeringen af APIen noget nemmere. Jeg nåede at sidde et par dage, sammen med Lennart, for at implementere kommunikationen fra den digital sensor til PSoC, men uden meget succes. 
Da den analoge sensor udsendte en PWM og ikke en DC-spændingen, som oftest er tilfældet, var det nødvendigt at for omdannet dette signal til en fast værdi. Dette blev løst med et simpelt 2. ordens lavpasfilter. 

