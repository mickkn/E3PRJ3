%Konklusion JS
\section{Jakob Schmidt}

3. semesterprojekt har for mig været udfordrende fagligt, specielt den serielle kommunikation har været en udfordring. I starten var der valgt en digital sensor, til datamåling af temperatur og fugtighed, men for at få en analog komponent med i projektet endte valget på SHT21p. Dette gjorde implementeringen af APIen noget nemmere. Jeg nåede at sidde et par dage, sammen med Lennart, for at implementere kommunikationen fra den digitale sensor til PSoCen, men uden meget succes. 
Da den analoge sensor udsendte en PWM og ikke en DC-spændingen, som oftest er tilfældet, var det nødvendigt at for omdannet dette signal til en fast værdi. Dette blev løst med et simpelt 2. ordens lavpasfilter. 

Det har været en mærkbar fordel, at vi har kunne beholde den gamle projektgruppen fra 2. semester. Det har betydet at vi hurtigt kom ind i de gamle roller, og at arbejdesprocessen gik glidende fra en start af. Gruppen har dog fået lidt ændringer i form af et nyt medlem og et gammelt medlem der måtte forlade gruppen, men begge ændringer forgik i ro og mag.

Vi har i år været i noget bedre tid end sidste år, hvilket har været tydelig mærkbart de sidste dage op mod afleveringen. Vi har haft god tid, og har holdt dagene korte og overskuelige, hvor vi sidste år havde 2 rigtig lange dage til sen aften, for at blive færdig til tiden. 
Vi har desuden tid til at få lavet en alternativ løsning til pumpen.

Personligt er jeg godt tilfreds med slutproduktet. Det virker efter hensigten og vi kan se at der sker noget. Vi kunne sagtens have lavet det mere komplekst og udfordrende, men da vi havde et stort overtal af HW-folk er det ikke blevet tilfældet. Det store overtal af HW-folk har været mærkbart, og en mere ligelig fordeling havde været at fortrække, specielt da rigtig store dele af projektet består af software. 


