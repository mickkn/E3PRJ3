\subsection{SPI (MK PO)}

Der udarbejdes et Kernemodul, en driver og en API for SPI kommunikationen for Master og en SPI Handler til Enheden. Disse er beskrevet i dette afsnit.

\subsubsection*{Kernemodul og driver}

Kernemodulet til SPI kommunikationen sørger for at oprette en fil i filsystemet på Master. Det opretter en systemfil/systemnode der kan skrive og læses fra. Dette er praktisk da man kan skrive metoder der lige netop kan læse og skrive til en fil i filsystemet i Linux operativsystemet. Kernemodulet er genbrugt fra HAL Exercise 7\footnote{Hardware abstraktioner. Exercise 7: LDD with SPI. Øvelse med SPI-kommunikation} og driveren er en videreudbygning af driveren fra samme øvelse, så denne kommunikere med en \verb+CHAR+ per transmission.

\subsubsection*{API}

SPI APIen er en klasse skrevet i \verb+C+++ der indeholder metoder til at kommunikere via SPI med Driveren og det dertilhørende kernemodul. Det består af en række metoder beskrevet i Projektdokumentationen afsnit 5.6.2 og afsnit 6.13. 

Alle metoderne bruger en filnode \verb+/dev/spi_dev+, som den skriver og læser fra. Alle metoderne er udarbejdet sammen med en SPI Handler på Enhed(PSoC4) der kan håndtere de forskellige kommandoer fra tabel \ref{tabel:SWProtokol-kommandoer}.

\subsubsection*{Handler}

Handleren er som sagt udarbejdet sammen med APIen. Det er en interrupt-rutine der handler på kommandoer i tabel \ref{tabel:SWProtokol-kommandoer} med en switch, der kan kalde metoder eller læse data ved en given kommando. Den er beskrevet yderligere i Projektdokumentationen afsnit 5.6.3 og implementeret i bilag under \verb+SW/Enhed/Kildekode/spi_handler+