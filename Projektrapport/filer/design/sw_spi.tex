\subsection{SPI (MK PO)}

Der udarbejdes et Kernemodul, en driver og en API for SPI kommunikationen for Master og en SPI Handler til Enheden. Disse er beskrevet i dette afsnit. For yderligere information se Projektdokumentation afsnit 7.6 SPI.

\subsubsection*{Kernemodul og driver}

Kernemodulet til SPI kommunikationen skal sørge for at oprette en fil i filsystemet på Master. Det skal oprette en systemfil/systemnode der kan skrive og læses fra. Dette er praktisk da man kan skrive metoder der lige netop kan læse og skrive til en fil i filsystemet i Linux operativsystemet. Kernemodulet kan genbruges fra HAL Exercise 7\footnote{Hardware abstraktioner. Exercise 7: LDD with SPI. Øvelse med SPI-kommunikation} og driveren kan videreudvikles ud fra driveren i samme øvelse, så denne kommunikere med en \verb+CHAR+ per transmission.

\subsubsection*{API}

SPI APIen skal være en klasse skrevet i \verb+C+++ der skal indeholde metoder til at kommunikere via SPI med Driveren og det dertilhørende kernemodul. Det skal bestå af en række metoder beskrevet i Projektdokumentationen afsnit 7.6.2 SPI API og afsnit 6.2.3 Klassebeskrivelser. 

Alle metoderne skal bruge en filnode \verb+/dev/spi_dev+, som den skriver og læser fra. Alle metoderne udarbejdes sammen med en SPI Handler på Enhed (PSoC4) der kan håndtere de forskellige kommandoer fra tabel \ref{tabel:SWProtokol-kommandoer}.

\subsubsection*{Handler}

Handleren skal som sagt udarbejdes sammen med APIen. Det skal være en interrupt-rutine der handler på kommandoer i tabel \ref{tabel:SWProtokol-kommandoer} med en switch, der kan kalde metoder eller læse data ved en given kommando.