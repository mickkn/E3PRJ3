\subsection{Strømforsyning og tilslutningsprint (MK PO SK)}

Der er blevet udarbejdet et tilslutningsprint med tilhørende strømforsyning for at gøre systemet uafhængig af laboratoriet, se figur \ref{fig:PSU_connections}. Strømforsyningen er en 230 VAC / 12 VDC switch-mode-konverter.

\figur{1}{tilslutningsprint}{Kredsløb over samlet spændingsforsyning (3,3 V og 5 V) og tilslutninger til PSoC4}{fig:PSU_connections}

Tilslutningsprintet forsyner PSoC4, Sprinkler-relæ, PIR-sensor og FT-sensor. LM7805 og LM317 er spændingsregulatorer som er blevet brugt til henholdsvis at regulere spændingen ned til 5 VDC og 3,3 VDC. I projektdokumentation afsnit 5.5.1, Strømforsyningen, kan beregninger ses for udgangsspænding og afsat effekt i spændingsregulatoren. Effektudregningen viser om det er nødvendigt at køle på spændingsregulatoren, dette er vigtigt at tage højde for, da komponenten ellers kan tage skade. Effektudregningen viser at det kun er nødvendigt med køling på LM7805, men der er fortaget køling på begge regulatorer alligevel. Det skyldes at det altid er en god ide at køle på en komponent, hvis der er mulighed for det, dette sikre bl.a. en længere levetid og berøringssikkerhed.

   