%SPI
\subsection{SPI (MK PO SK)}

Serial Parrallel Interface (SPI) er en måde at lave hurtig seriel dataudveksling på. SPI er udviklet af Motorola og fungerer ved at man, som regel, har en enkelt masterenhed der styrer flere slaveenheder. Ved SPI er der er ingen fejl-check og adressering af flere enheder kan være HW krævende. I EasyWater8000 projektet er der SPI kommunikation imellem Master (Devkit8000) og Enhed (PSoC4), det giver muligheden for at overføre flere data på samme tid imellem disse to.

\begin{figure}[H] \centering
{\includegraphics[width=0.7\textwidth]{billeder/design_spi_master_slave}}
\caption{SPI-bus}
\label{fig:design_spi_bus}
\raggedright
\end{figure}

På figur \ref{fig:design_spi_bus}\footnote{Kilde: Wikipedia om SPI} ses en typisk opkobling imellem Master og Slave, det kræver 4 tråde.

\begin{itemize}
 	\item SS står for "Slave select".
 	\item MISO står for "Master in slave out".
 	\item MOSI står for "Master out slave in". 
	\item SCLK står for "Serial clock" 
\end{itemize}

SPI-kommunikation er baseret på skifteregister princippet, som ses på figur\ref{fig:design_spi_register} \footnote{Kilde: SPI Slide i GFV v/ Michael Alrøe}. Der vælges hvilken slave der ønskes at skrives til ved slave select(SS), derefter shiftes et 8-bit register 1 bit ad gangen. Serial clock er den clock der sørger for at shiftningen af bits sker korrekt, clockfrekvensen må ikke overstige grænsen for hvad enhederne kan håndtere.

\begin{figure}[H] \centering
{\includegraphics[width=0.8\textwidth]{billeder/design_spi_register}}
\caption{SPI-register}
\label{fig:design_spi_register}
\raggedright
\end{figure}  

En ofte anvendt kommunikation imellem master og slave foregår ved at master sætter SS til 0. Den fortæller nu til slaven at den er klar til 
dataoverførsel. Nu sender masteren en bit over til slaven og påbegynder en half-duplex data transmission, samtidigt sender slaven en bit over til masteren. Denne process fortsætter indtil masteren har sendt alle sine bits, derefter stopper masteren med at toggle på clocken og slave enheder bliver frigivet.

SPI kan køre både full- og halfduplex, dvs. at der kan være 2-vejs kommunikation ved full-duplex (flere masterenheder) eller 2-vejs kommunikation ved half-duplex, hvor en masterenhed styrer/skriver til en eller flere slaveenheder. I EasyWater8000 er der behov for at der kommunikeres 2 veje, men da Enheden fungerer som slave, kan den ikke selvstændigt sende til Masteren, men behøver et kald, om at nu skal den sende information. Derfor er det half-duplex der bruges.

\subsubsection*{Kernemodul og driver}

Kernemodulet til SPI kommunikationen sørger for at oprette en fil i filsystemet på Master. Det opretter en systemfil/systemnode der kan skrive og læses fra. Dette er praktisk da man kan skrive metoder der lige netop kan læse og skrive til en fil i filsystemet i Linux operativsystemet. Kernemodulet er genbrugt fra HAL Exercise 7\footnote{Hardware abstraktioner. Exercise 7: LDD with SPI. Øvelse med SPI-kommunikation} og driveren er en videreudbygning af driveren fra samme øvelse, så denne kommunikere med en \verb+CHAR+ per transmission.

\subsubsection*{API}

SPI APIen er en klasse skrevet i \verb+C+++ der indeholder metoder til at kommunikere via SPI med Driveren og det dertilhørende kernemodul. Det består af en række metoder beskrevet i Projektdokumentationen afsnit 5.6.2 og afsnit 6.13. 

Alle metoderne bruger en filnode \verb+/dev/spi_dev+, som den skriver og læser fra. Alle metoderne er udarbejdet sammen med en SPI Handler på Enhed(PSoC4) der kan håndtere de forskellige kommandoer fra tabel \ref{tabel:SWProtokol-kommandoer}.

\subsubsection*{Handler}

Handleren er som sagt udarbejdet sammen med APIen. Det er en interrupt-rutine der handler på kommandoer i tabel \ref{tabel:SWProtokol-kommandoer} med en switch, der kan kalde metoder eller læse data ved en given kommando. Den er beskrevet yderligere i Projektdokumentationen afsnit 5.6.3 og implementeret i bilag under \verb+SW/Enhed/Kildekode/spi_handler+