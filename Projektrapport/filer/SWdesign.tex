\section{Softwaredesign}

\subsection{Applicationsmodel}
I softwaredesign fasen lavede vi applications modeller for de forskellige use cases for at få et overblik over hvilke metoder og funktioner vi skulle bruge. Applications modellen består af, at man først laver sekvensdiagrammer for hver use case. Hvis man finder det nødvendigt for at få et overblik kan man også lave state machine over hver use case. Dette følte vi dog ikke vi havde brug for så det step er sprunget over.

3. step i applicationsmodellen er at få alle metode kald i sekvensdiagrammerne, over i et klassediagram. Dette klassediagram er ment som et overblik som kan benyttes når man går igang med den endelige klassebeskrivelse.

I klassebeskrivelsen beslutted der så hvilke datatyper man vil have i de forskellige klasser og hvad de enkelte klasser hver i sær har ansvar for. Når klassebeskrivelsen er udarbejdet vil man så komme frem til et statisk klassediagram som er endeligt.

\figur{1}{a_uc2}{Sekvensdiagram fra applicationsmodel}{fig:devkit_a_uc2}

I figur \ref{fig:devkit_a_uc2} kan man se et eksempel på et sekvensdiagram udviklet i vores applicationsmodel. Sekvensdiagrammet er lavet ud fra use case 2: Konfig. Den viser alle de involverede klasser og metode kaldene imellem dem.

På figur \ref{fig:devkit_class} kan man se vores conceptuelle klassediagram som er udviklet i applicationsmodellen.

\figur{1}{sw_class_devkit}{klassediagram fra applicationsmodellen}{fig:devkit_class}

Applicationsmodellen er benyttet både til softwaredesign af koden til devkit8000 og PSoC4. Det er dog 2 forskellige frameworks der arbejdes i men dette har ingen indflydelse på vores applicationsmodel da vi beskæftigere os med conceptet i hvordan koden skal fungere. På PSoC4 kunne vi ikke arbejde med C++ og objekt orienteret programmering som vores dokumentation lægger op til. Dette løses ved at fortolke klassebeskrivelsen iht. UML-Light \footnote{T-133 UML-Light af Finn Overgaard Hansen. Benyttet i forbindelse med I1ORPG(Objektorienteret programmering) på 1. semester}

\subsection{Klassebeskrivelser}

Efter applicationsmodellen blev klasserne beskrevet. Hver klasse har et overordnet ansvar. F.eks har controller klassen Config som vi så på figur \ref{fig:devkit_a_uc2} ansvaret for at styre UC2. Hver metode i klassen har således en beskrivelse af hvad metode skal gøre, hvilke parametre den modtager og hvad metoden returnere. I vores tilfælde aftalte vi at alle metoder skulle returnere en integer for at vi kunne lave fejl-tjek på dem. Det vil sige at når metoderne skulle modulere noget data som den fik tilsendt så skulle det være en pointer den modtog.


%SPI
\subsection{SPI (MK PO)}

Der udarbejdes et kernemodul, en driver og en API for SPI-kommunikationen for Master og en SPI-handler til Enheden. Disse er beskrevet i dette afsnit. For yderligere information se projektdokumentation afsnit 7.6, SPI.

\subsubsection*{Kernemodul og driver}

Kernemodulet til SPI-kommunikationen skal sørge for at oprette en fil i filsystemet på Master. Det skal oprette en systemfil/systemnode der kan skrive og læses fra. Dette er praktisk da der kan skrives metoder der lige netop kan læse og skrive til en fil i filsystemet i Linux-operativsystemet. Kernemodulet kan genbruges fra HAL Exercise 7\footnote{Hardware abstraktioner. Exercise 7: LDD with SPI. Øvelse med SPI-kommunikation} og driveren kan videreudvikles ud fra driveren i samme øvelse, så denne kommunikerer med én \verb+char+ per transmission.

\subsubsection*{API}

SPI-APIet skal være en klasse, skrevet i \verb-C++-, der skal indeholde metoder til at kommunikere via SPI med driveren og det dertilhørende kernemodul. Det skal bestå af en række metoder beskrevet i projektdokumentationen afsnit 7.6.2m SPI-API og afsnit 6.2.3, Klassebeskrivelser. 

Alle metoderne skal bruge en filnode \verb+/dev/spi_dev+, som den skriver og læser fra. Alle metoderne udarbejdes sammen med en SPI-handler på Enheden (PSoC4) der kan håndtere de forskellige kommandoer fra tabel \ref{tabel:SWProtokol-kommandoer}.

\subsubsection*{Handler}

Handleren skal som sagt udarbejdes sammen med APIet. Det skal være en interrupt-rutine der handler på kommandoer i tabel \ref{tabel:SWProtokol-kommandoer}, i afsnit \ref{head:kommunikationsprotokol}, med et switch-case, der kan kalde metoder eller læse data ved en given kommando.