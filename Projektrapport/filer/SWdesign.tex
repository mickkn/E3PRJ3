\section{Softwaredesign (BS JC)}

\subsection{Applicationsmodel}
I softwaredesign-fasen laves applikationsmodeller for de forskellige use cases, for at få et overblik over hvilke metoder og funktioner der skal bruges. Applikationsmodellen består af, at man først laver sekvensdiagrammer for hver use case. Hvis det finders nødvendigt for at få et overblik kan der også laves state machine-diagrammer over hver use case. Dette er ikke vurderet nødvendigt hvorfor det er udeladt.

Næste step i applikationsmodellen er at få alle metode-kald i sekvensdiagrammerne, over i et klassediagram. Dette klassediagram er ment som et overblik, som kan benyttes når der arbejdes med den endelige klassebeskrivelse.

I klassebeskrivelsen besluttes der så hvilke datatyper der ønskes i de forskellige klasser og hvad de enkelte klasser hver i sær har ansvar for. Når klassebeskrivelserne er udarbejdet vil resultatet være et, endeligt, statisk klassediagram. Se dokumentationen afsnit 6.2 Applikationsmodeller for detaljer.

\figur{1}{a_uc2}{Sekvensdiagram fra applikationsmodel}{fig:devkit_a_uc2}

I figur \ref{fig:devkit_a_uc2} ses et eksempel på et sekvensdiagram udviklet i applikationsmodellen. Sekvensdiagrammet er lavet ud fra use case 2: Konfig. Den viser alle de involverede klasser og metode-kaldene imellem dem.

På figur \ref{fig:devkit_class} ses det konceptuelle klassediagram som er udviklet i applikationsmodellen.

\figur{1}{sw_class_devkit}{klassediagram fra applikationsmodellen}{fig:devkit_class}

Applikationsmodellen er benyttet både til softwaredesign af koden til Devkit8000 og PSoC4. Det er dog to forskellige frameworks der arbejdes i, men dette har ingen indflydelse på applikationsmodellen, da denne beskæftiger sig med konceptet i hvordan koden skal fungere og ikke implementeringen af denne. På PSoC4 er det ikke muligt at arbejde i \verb-C++- og med objekt orienteret programmering som dokumentationen lægger op til. Dette løses ved at fortolke klassebeskrivelsen iht. UML-Light\footnote{T-133 UML-Light, Finn Overgaard Hansen, 2013, Ingeniørhøjskolen Aarhus Universitet}.

\subsection{Klassebeskrivelser}

Efter applikationsmodellen blev klasserne beskrevet. Hver klasse har et overordnet ansvar. F.eks har controller-klassen, \verb+Config+ som blev beskrevet på figur \ref{fig:devkit_a_uc2}, ansvaret for at styre UC2. Hver metode i klassen har således en beskrivelse af hvad metoden skal gøre, hvilke parametre den modtager og hvad metoden returnerer. Over hele projektet er det aftalt, at alle metoder skal returnere en integer for at der kunne laves fejl-tjek på dem. Det vil sige at når metoderne skal modulere noget data som denne fik tilsendt, så skal det være en pointer den modtager.

Efter klassebeskrivelserne er det konceptuelle klassediagram, med alle de nye medlemsdata, parametre og datatyper der benyttes i de pågældende klasser, blevet opdateret. Det nye, opdateret, klassediagram kan ses på figur \ref{fig:sw_class_static} i softwareimplementeringen, afsnit \ref{head:sw_impl}. Se alle klassebeskrivelser i dokumentationen afsnit 6.2.3 Klassebeskrivelser.

%SPI
\subsection{SPI (MK PO)}

Der udarbejdes et kernemodul, en driver og en API for SPI-kommunikationen for Master og en SPI-handler til Enheden. Disse er beskrevet i dette afsnit. For yderligere information se projektdokumentation afsnit 7.6, SPI.

\subsubsection*{Kernemodul og driver}

Kernemodulet til SPI-kommunikationen skal sørge for at oprette en fil i filsystemet på Master. Det skal oprette en systemfil/systemnode der kan skrive og læses fra. Dette er praktisk da der kan skrives metoder der lige netop kan læse og skrive til en fil i filsystemet i Linux-operativsystemet. Kernemodulet kan genbruges fra HAL Exercise 7\footnote{Hardware abstraktioner. Exercise 7: LDD with SPI. Øvelse med SPI-kommunikation} og driveren kan videreudvikles ud fra driveren i samme øvelse, så denne kommunikerer med én \verb+char+ per transmission.

\subsubsection*{API}

SPI-APIet skal være en klasse, skrevet i \verb-C++-, der skal indeholde metoder til at kommunikere via SPI med driveren og det dertilhørende kernemodul. Det skal bestå af en række metoder beskrevet i projektdokumentationen afsnit 7.6.2m SPI-API og afsnit 6.2.3, Klassebeskrivelser. 

Alle metoderne skal bruge en filnode \verb+/dev/spi_dev+, som den skriver og læser fra. Alle metoderne udarbejdes sammen med en SPI-handler på Enheden (PSoC4) der kan håndtere de forskellige kommandoer fra tabel \ref{tabel:SWProtokol-kommandoer}.

\subsubsection*{Handler}

Handleren skal som sagt udarbejdes sammen med APIet. Det skal være en interrupt-rutine der handler på kommandoer i tabel \ref{tabel:SWProtokol-kommandoer}, i afsnit \ref{head:kommunikationsprotokol}, med et switch-case, der kan kalde metoder eller læse data ved en given kommando.