\subsection{Kommunikationsprotokol (MK, PO)}
Kommunikationsprotokollen for SPI kommunikationen er herunder beskrevet. Tabel \ref{tabel:SWProtokol-kommandoer} viser de mulige kommandoer for SPI kommunikationen. 

Opsætningen er som følger:

\begin{itemize}
  \item Hastighed: 1 MHz
  \item SPI mode: 0 (CPOL 0 - CPHA 0)
  \item Antal bits: 1 char pr. transmission
\end{itemize}

PSoC4 kan køre 8 MHz, men der er ikke behov for så høj hastighed. Stabiliteten blev forbedret væsentligt ved at vælge en lavere hastighed. 
\newline SPI-mode = 0 - er valgt på baggrund af default indstillinger. 
\newline CPOL = 0 - vil sige at clocken er lav når den er passiv (aktiv-høj). 
\newline CPHA = 0 - vil sige at data udlæses på rising-edge. 
\newline Der transmitteres en karakter pr. transmission dvs. 8 bits.

\begin{table}[H]
\caption{Kommandoer for SPI-kommunikation}
\centering
\begin{tabular}{|c|c|l|c|}
\hline 
\textbf{ASCII} & \textbf{HEX} & \textbf{Funktion} \\ 
\hline 
'A' & 0x41 & Aktiver Enhed \\ 
\hline 
'D' & 0x44 & Deaktiver Enhed \\ 
\hline 
'P' & 0x50 & Parametre sendes til Enhed \\
\hline 
'V' & 0x56 & Verificer Enhed i systemet \\ 
\hline
'L' & 0x4c & Forespørg logdata fra Enhed \\ 
\hline
'C' & 0x4c & Write buffer og clearing af tx-buffer  \\
\hline
'R' & 0x4c & Læsning af data fra Enhed \\
\hline
\end{tabular}
\label{tabel:SWProtokol-kommandoer}
\end{table} 


For yderligere specifikation af de enkelte kommandoer se dokumentationen afsnit 4.3.5 Kommunikationsprotokol.