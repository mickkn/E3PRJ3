\chapter{Kravspecifikation}

% Indledning

\section{Indledning}
Gruppen vil udvikle et system til vanding af golfbaner. I forbindelse med stigende temperature bliver det mere kritisk at vandingen af golfbaner sker på de rigtige tidspunkter for at holde banen spilbar. For at spare på resourcerne er det også kritisk ikke at spilde store mængder vand på vanding af områder som ikke trænger til det.

Med et system af intelligente enheder der kan arbejde autonomt, men som modtager indstillinger for vandingen fra en masterenhed, herved kan man spare arbejdstid for greenkeeperen og vandingen sker kun når det er nødvendigt.

Normalt vil man placere en af disse enheder ved hvert hul på golfbanen og lave et netværk af sensorer lokalt til denne enhed. Den kan dermed overvåge området og vande om nødvendigt. Alle enhederne forbindes til et netværk som er koblet sammen med Master. Grænseværdierne, for f.eks. jordfugtigheden, der afgør hvornår enheden skal påbegynde vanding kommer fra Master og styres igennem denne af greenkeeperen. 

\begin{figure}[ht] \centering
\fbox{\includegraphics[height=4.1cm]{filer/indledning/billeder/hul_med_sprinkler}}
\caption{Systemoverblik}
\label{fig:systemoverblik}
\end{figure}

FIGURBESKRIVELSE MANGLER!!!

Oversigt over blokkene i systemet kan ses i figur \ref{fig:bloksystemoverblik}.
Bemærk at der er muligt at forbinde flere af hver type sensor til én enhed. Hvis altså et hul på en golfbane kræver tre temperatursensorer, kobles disse blot på den samme enhed.

\begin{figure}[ht] \centering
\fbox{\includegraphics[height=10cm]{filer/indledning/billeder/vandingssystem}}
\caption{Blokoverblik af system}
\label{fig:bloksystemoverblik}
\end{figure}

FIGUREN HEROVER SKAL DISKUTERES PÅ MØDE!

Systemet laves med en PSoC som Enhed, altså den del der håndterer data fra sensorene mv.
Devkit8000 fungerer som Master og er brugerinterface.

En Enhed består altså af:
\begin{enumerate}
\item Fugtighedssensor (evt. flere)
\item Temperatursensor (evt. flere)
\item Bevægelsessensor (evt. flere)
\item Sprinkler (evt. flere)
\item PSoC controller board
\end{enumerate}

Fugt- og temperatursensorerne registrerer banens behov for vanding. Bevægelsessensoren registrerer om der er spillere på det pågældende hul. Sprinkleren sørger for vandingen når der skal vandes. Denne skjules nede i græsset og kommer op når vandingen skal være aktiv.
PSoC controller-boardet bliver hjernen i Enheden. Denne styrer kommunikationen til Master og holder styr på de generelle funktioner for Enheden så som udlæsning af data, aktivering af sprinkler mv.

% Aktører

\section{Aktører}
Her beskrives systemets aktøre. Disse vil blive refereret i de efterfølgende usecase-beskrivelser.

%% !!! Aktør kontekst diagram !!!
%
%\fbox{\includegraphics[width=0.9\textwidth]{billeder/diagrammer/Kontekst_Diagram}}
%\caption{Kontekst diagram}
%\label{lab:kontekstdiagram}
%\end{figure}

\begin{table}[!htbp] \centering
	\begin{tabular}{|p{2.5cm}|p{11.5cm}|}
	\hline
		\textbf{Aktør navn} & \textbf{Beskrivelse} \\\hline
		Bruger & Bruger-aktøren vil normalt være greenkeeperen. Det er vedkommende som kontrollere og betjener systemet. (Primær) \\\hline

		Omgivelser & Almene omgivelser, som har indflydelse på systemets sensorer. Det være sig temperatur, fugtighed og bevægelser i områderne omkring systemet. (Sekundær) \\\hline
	\end{tabular}
\end{table}

Ud over de nævnte aktører bruges også navnene på nogle af systemets dele.


% Usecases

\section{Usecases}\label{header:usecases}

Her følger en dybere beskrivelse af systemets opbygning og måde at virke på. Dette gøres med fulde usecase beskrivelser hvor systemets virkning er beskrevet i detaljer.

\begin{figure}[!htbp] \centering
\vspace*{\fill}
\includegraphics[width=\textwidth]{filer/kravspec/visio/Usecase_Diagram}
\caption{Usecase diagram}
\label{lab:usecasediagram}
\vspace*{\fill}
\end{figure}

%% !!! Use case diagram !!!
%
%\begin{figure}[!htbp] \centering
%\section{Usecases}
%\vspace*{\fill}
%\includegraphics[width=\textwidth]{billeder/diagrammer/Usecase_Diagram}
%\caption{Usecase diagram}
%\label{lab:usecasediagram}
%\vspace*{\fill}
%\end{figure}

% UC1: Tilføj/fjern enhed

\subsection{UC1: Tilføj/fjern enhed}
\begin{center} \centering \label{UC1}
	\begin{longtable}{|p{5cm}|p{9cm}|}  %% Longtable = forsætter på næste side
	\hline
		\multicolumn{2}{|l|}{\textbf{UC1: Tilføj\slash fjern enhed}} \\\hline %% HUSK UCECASE NUMMER + NAVN
		\endfirsthead
		
		\multicolumn{2}{l}{...fortsat fra forrige side} \\ \hline %% Til LONGTABLE
		\multicolumn{2}{|l|}{\textbf{UC1: Tilføj\slash fjern enhed}} \\\hline %% HUSK UCECASE NUMMER + NAVN
		\endhead	
		
		\textbf{Mål}								&Bruger kan tilføje eller fjerne enheder fra systemet			\\\hline
		\textbf{Initialisering}					&Bruger														\\\hline
		\textbf{Aktører og Stakeholders}			&Bruger(Primær)												\\\hline 
		\textbf{Referencer}						&Ingen														\\\hline
		\textbf{AASH}							&1															\\\hline
		\textbf{Efterfølgende tilstand}			&Ønsket Enhed er tilføjet eller fjernet fra systemet.		\\\hline
		\textbf{Hovedforløb}					
			&\begin{enumerate}
	
				\item Bruger vælger ''Tilføj/fjern enhed'' i hovedmenu
				
				\item \label{uc1valg} Bruger vælger ''Tilføj enhed''
				
				\item En liste af opsatte enheder præsenteres på skærmen				
				
				\begin{enumerate}
					\item \label{uc1indtast} Master beder bruger om at indtaste informationer
					
					\item \label{uc1indtast_fejl} Bruger indtaster \textcolor{red}{navn og adresse} på Enhed
					
						\textbf{[Undtagelse \ref{uc1indtast_fejl}.a]} \newline
						Indtastede værdier er ikke gyldige
					
					\item Master tilføjer Enhed til systemet med default parametre
					
					\item Enhed forbindes til kommunikationsnetværket
					
					\item \label{uc1verif} Master verificerer forbindelsen til Enhed og sender dato og tidspunkt
						
					\textbf{[Undtagelse \ref{uc1verif}.a]} \newline
					Enheden kan ikke verificeres
					
					\item Enhed gemmer dato og tidspunkt
				\end{enumerate}

				\item Bruger vælger ''Fjern enhed''

				\item En liste af opsatte enheder præsenteres på skærmen

				\begin{enumerate}
					
					\item Bruger indtaster adresse på Enhed
					
					\item Master deaktivere Enhed
					
					\item Master sletter Enhed fra systemet
				
				\end{enumerate}
				
				\item Master opdaterer liste med opsatte enheder
				
				\item Bruger kan returnere til hovedmenu eller opsætte ny enhed (Gå til UC1.\ref{uc1valg})
			\end{enumerate}\\\hline
		\textbf{Undtagelser}
			&\begin{enumerate}[label=\ref{uc1indtast}.a]
				
				\item Master viser fejlbesked omkring ugyldige værdier
				
					\subitem Gå til UC1.\ref{uc1indtast}
			\end{enumerate}
			
			\begin{enumerate}[label=\ref{uc1verif}.a]
				
				\item Master viser fejlbesked angående verificering af enheden
				
				\item Bruger kan forsøge igen (Gå til UC1.\ref{uc1verif} eller afbryde)

			\end{enumerate}														\\\hline
	\end{longtable} 
\end{center}

%% TIPS:
%% LABEL TIL PUNKT: \label{labelnavn}
%% REFERENCE: \ref{labelnavn}

% UC2: Config

\subsection{UC2: Konfig}
\begin{center} \centering \label{UC2}
	\begin{longtable}{|p{5cm}|p{9cm}|}  %% Longtable = forsætter på næste side
	\hline
		\multicolumn{2}{|l|}{\textbf{UC2: Konfig}} \\\hline %% HUSK UCECASE NUMMER + NAVN
		\endfirsthead
		
		\multicolumn{2}{l}{...fortsat fra forrige side} \\ \hline %% Til LONGTABLE
		\multicolumn{2}{|l|}{\textbf{UC2: Konfig}} \\\hline %% HUSK UCECASE NUMMER + NAVN
		\endhead	
		
		\textbf{Mål}								&Ændre default parametre på en Enhed			\\\hline
		\textbf{Initialisering}					&Bruger							\\\hline
		\textbf{Aktører og Stakeholders}			&Bruger(Primær)					\\\hline
		\textbf{Referencer}						&Ingen							\\\hline
		\textbf{AASH}							&1								\\\hline
		\textbf{Efterfølgende tilstand}			&Ingen							\\\hline
		\textbf{Hovedforløb}					
			&\begin{enumerate}
	
				\item Bruger vælger ''Konfig'' i hovedmenu
				
				\item \label{uc2afbryd}Bruger vælger den Enhed, der ønskes omkonfigureret.\newline
				\textbf{[Undtagelse \ref{uc2afbryd}.a]} \newline
					Bruger vælger afbryd
				
				\item \label{uc2afbryd2}Bruger ændrer parametre på den valgte Enhed.
				\textbf{[Undtagelse \ref{uc2afbryd2}.a]} \newline
					Der indtastes ugyldige værdier	
									
				\item \label{uc2afbryd3} Bruger vælger ''Gem''\newline
				\textbf{[Undtagelse \ref{uc2afbryd3}.a]} \newline
					Bruger vælger afbryd
				
				\item Master sender de nye indstillinger til den valgte Enhed. 
			
	
			\end{enumerate}\\\hline
		\textbf{Undtagelser}
			&\begin{enumerate}[label=\ref{uc2afbryd}.a]
				
				\item Bruger vælger ''afbryd''
				
					\subitem Skærmen viser hovedmenu
			\end{enumerate}
			
			\begin{enumerate}[label=\ref{uc2afbryd2}.a]
				
				\item Der indtastes ugyldige værdier	
				
					\subitem Skærmen viser fejlbesked
			\end{enumerate}			
			
			\begin{enumerate}[label=\ref{uc2afbryd3}.a]
				
				\item Bruger vælger ''afbryd''
				
				\subitem Skærmen viser Konfig-menu
				

			\end{enumerate}														\\\hline
	\end{longtable}
\end{center}

%% TIPS:
%% LABEL TIL PUNKT: \label{labelnavn}
%% REFERENCE: \ref{labelnavn}

% UC3: Aktiver/deaktiver

\subsection{UC3: Aktiver/deaktiver}
\begin{center} \centering
	\begin{longtable}{|p{6cm}|p{8cm}|}  %% Longtable = forsætter på næste side
	\hline
		\multicolumn{2}{|l|}{\textbf{UC3: Planlagt vanding}} \\\hline %% HUSK UCECASE NUMMER + NAVN
		\endfirsthead
		
		\multicolumn{2}{l}{...fortsat fra forrige side} \\ \hline %% Til LONGTABLE
		\multicolumn{2}{|l|}{\textbf{UC3: Planlagt vanding}} \\\hline %% HUSK UCECASE NUMMER + NAVN
		\endhead	
		
		\textbf{Mål}								&Udføre planlagt vanding			\\\hline
		\textbf{Initialisering}					& Bruger vælger "Planlagt vanding"			\\\hline
		\textbf{Aktører og Stakeholders}			&Bruger(primær)			\\\hline
		\textbf{Referencer}						&Ingen			\\\hline
		\textbf{Antal af samtidige hændelser}	&1 			\\\hline
		\textbf{Forudsætning}					&At valgte enhed er aktiv			\\\hline
		\textbf{Efterfølgende tilstand}			&Autonom tilstand			\\\hline
		\textbf{Hovedforløb}					
			&\begin{enumerate}
	
				\item Bruger vælger planlagt vanding i UI    %%Punkt 1
				
				\item Bruger vælger "Opret ny" [Undtagelse 1.a][Undtagelse 1.b]
				
				\item Bruger indstiller ønsket dato
				
				\item Bruger indstiller ønsket ON tidsrum
				
				\item Bruger indstiller ønsket OFF tidsrum
				
				\item Bruger indstiller ønsket vandingsmængde i ON tidsrummet %%Punkt 2
				
				\item Bruger gemmer planlagt vanding%%Etc.
				
				\item Bruger vender tilbage til hovedmenu
	
			\end{enumerate}\\\hline
		\textbf{Undtagelser}					
			&\begin{enumerate}
				\item Undtagelse 1.a : Bruger vælger "Vis planlagte vandinger"
					\begin{enumerate}
						\item Brugeren vises planlagte vandinger 
						\item Bruger trykker tilbage
						\item Bruger vender tilbage til hovedmenu	
					\end{enumerate}	
				\item [Undtagelse 1.b] : Brugeren vælger "Fjern planlagt vanding"
					\begin{enumerate}
						\item Brugeren vises en liste af planlagte vandinger.
						\item Brugeren indstaster det vandings ID som ønskes slettet
						\item Den planlagte vanding slettes	
						\item Bruger vender tilbage til hovedmenu
					\end{enumerate}	
			\end{enumerate}\\\hline
	\end{longtable}
	\label{UC3} 
\end{center}

%% TIPS:
%% LABEL TIL PUNKT: \label{labelnavn}
%% REFERENCE: \ref{labelnavn}

% UC4: Indsamle data

\subsection{UC4: Indsamle data}
\begin{center} \centering \label{UC4}
	\begin{longtable}{|p{5cm}|p{9cm}|}  %% Longtable = forsætter på næste side
	\hline
		\multicolumn{2}{|l|}{\textbf{UC4: Config}} \\\hline %% HUSK UCECASE NUMMER + NAVN
		\endfirsthead
		
		\multicolumn{2}{l}{...fortsat fra forrige side} \\ \hline %% Til LONGTABLE
		\multicolumn{2}{|l|}{\textbf{UC4: Config}} \\\hline %% HUSK UCECASE NUMMER + NAVN
		\endhead	
		
		\textbf{Mål}								&Konfigurer indstillinger på en enhed			\\\hline
		\textbf{Initialisering}					&Bruger åbner Config i UI		\\\hline
		\textbf{Aktører og Stakeholders}			&Bruger(Primær)					\\\hline
		\textbf{Referencer}						&Ingen							\\\hline
		\textbf{AASH}							&1								\\\hline
		\textbf{Forudsætning}					&Aktivt system					\\\hline
		\textbf{Efterfølgende tilstand}			&Ingen							\\\hline
		\textbf{Hovedforløb}					
			&\begin{enumerate}
	
				\item Bruger vælger Config i UI
				
				\item Bruger vælger den enhed brugeren ønsker at konfigurere.
				
				\item Bruger opsætter de ønskede indstillinger på den valgte enhed.
				
				\item Bruger færdiggøre indstillingen og vælger "Gem"
				
				\item Master uploader/programmere de nye indstillinger til den valgte enhed. 
	
			\end{enumerate}\\\hline
	\end{longtable}
\end{center}

%% TIPS:
%% LABEL TIL PUNKT: \label{labelnavn}
%% REFERENCE: \ref{labelnavn}

% UC5: Vanding

\subsection{UC5: Vanding}
\begin{center} \centering \label{UC5} 
	\begin{longtable}{|p{5cm}|p{9cm}|}  %% Longtable = forsætter på næste side
	\hline
		\multicolumn{2}{|l|}{\textbf{UC5: Vanding}} 		\\\hline %% HUSK UCECASE NUMMER + NAVN
		\endfirsthead
		
		\multicolumn{2}{l}{...fortsat fra forrige side} 	\\ \hline %% Til LONGTABLE
		\multicolumn{2}{|l|}{\textbf{UC5: Vanding}} 		\\\hline %% HUSK UCECASE NUMMER + NAVN
		\endhead	
		
		\textbf{Mål}								&At vande trængende områder	\\\hline
		\textbf{Initialisering}					&Ingen						\\\hline
		\textbf{Aktører og Stakeholders}			&Golfbane(Primær)			\\\hline
		\textbf{Referencer}						&UC4: Indsamle data 			\\\hline
		\textbf{AASH}							&1							\\\hline
		\textbf{Forudsætning}					&Aktiv Enhed					\\\hline
		\textbf{Efterfølgende tilstand}			&Aktiv						\\\hline
		\textbf{Hovedforløb}					
			&\begin{enumerate}
				
				\item Enhed læser fugt- og temperaturværdier fra UC4: Indsamle data
				
				\item \label{uc5sprinkler} Vanding startes på områder med for lave værdier
				
					\textbf{[Undtagelse \ref{uc5sprinkler}a]} Bevægelse registreret
	
			\end{enumerate}\\\hline

		\textbf{Undtagelser}
			&\begin{enumerate}[label=\ref{uc5sprinkler}a.]
			
				\item Sprinkler deaktiveres i 30 minutter og herefter gentages hovedforløb	
			
			\end{enumerate}\\\hline
	\end{longtable}
\end{center}

%% TIPS:
%% LABEL TIL PUNKT: \label{labelnavn}
%% REFERENCE: \ref{labelnavn}


% UC6: Tjek status

\subsection{UC6: Tjek status}
\begin{center} \centering
	\begin{longtable}{|p{6cm}|p{8cm}|}  %% Longtable = forsætter på næste side
	\hline
		\multicolumn{2}{|l|}{\textbf{UC6: Tjek status}} \\\hline %% HUSK UCECASE NUMMER + NAVN
		\endfirsthead
		
		\multicolumn{2}{l}{...fortsat fra forrige side} \\ \hline %% Til LONGTABLE
		\multicolumn{2}{|l|}{\textbf{UC6: Tjek status}} \\\hline %% HUSK UCECASE NUMMER + NAVN
		\endhead	
		
		\textbf{Mål}								&At kontrollere status på systemet			\\\hline
		\textbf{Initialisering}					&Bruger vælger "Tjek status"			\\\hline
		\textbf{Aktører og Stakeholders}			&Bruger(Primær)			\\\hline
		\textbf{Referencer}						&Ingen			\\\hline
		\textbf{Antal af samtidige hændelser}	&1			\\\hline
		\textbf{Forudsætning}					&Systemet er aktivt/tændt			\\\hline
		\textbf{Efterfølgende tilstand}			&Systemet viser hovedmenuen på devkit			\\\hline
		\textbf{Hovedforløb}					
			&\begin{enumerate}
	
				\item Bruger vælger "Tjek status"
				
				\item Status vises på devkit
				
				\item Bruger vælger "Tilbage"
	
			\end{enumerate}\\\hline
	\end{longtable}
	\label{UC6} 
\end{center}

%% TIPS:
%% LABEL TIL PUNKT: \label{labelnavn}
%% REFERENCE: \ref{labelnavn}

% UC7: Udskriv log

\subsection{UC7: Udskriv log}
\begin{center} \centering \label{UC7}
	\begin{longtable}{|p{5cm}|p{9cm}|}  %% Longtable = forsætter på næste side
	\hline
		\multicolumn{2}{|l|}{\textbf{UC7: Tjek status}} \\\hline %% HUSK USECASENUMMER + NAVN
		\endfirsthead
		
		\multicolumn{2}{l}{...fortsat fra forrige side} \\ \hline %% Til LONGTABLE
		\multicolumn{2}{|l|}{\textbf{UC7: Tjek status}} \\\hline %% HUSK USECASENUMMER + NAVN
		\endhead	
		
		\textbf{Mål}								&At kontrollere status på systemet			\\\hline
		\textbf{Initialisering}					&Bruger							\\\hline
		\textbf{Aktører og Stakeholders}			&Bruger(Primær)					\\\hline
		\textbf{Referencer}						&UC6: Send log					\\\hline
		\textbf{Antal af samtidige hændelser}	&1								\\\hline
		\textbf{Forudsætning}					&Aktiv Master				\\\hline
		\textbf{Efterfølgende tilstand}			&Hovedmenuen vises på Master			\\\hline
		\textbf{Hovedforløb}					
			&\begin{enumerate}
	
				\item Bruger vælger ''Tjek status''
				
				\item Status vises på Master
				
				\item Bruger vælger ''Tilbage''
	
			\end{enumerate}\\\hline
	\end{longtable} 
\end{center}

%% TIPS:
%% LABEL TIL PUNKT: \label{labelnavn}
%% REFERENCE: \ref{labelnavn}

%% Ikke-funktionelle krav

\section{Ikke-funktionelle krav}
% Ikke-funktionelle krav
\begin{enumerate}

\subsubsection*{Brugbarhed}
\item Opsætningen skal ske af autoriseret personale
\item UI skal kunne benyttes af bruger efter gennemlæst manual


\subsubsection*{Pålidelighed}
\item Levetid: 2 år uden hardware nedbrud
\item Software IDLE tid: minimum 1 måned uden restart


\subsubsection*{Ydeevne}
\item Master skal kunne håndtere minimum 18 enheder
\item Sprinkler skal kunne vande et areal afdækket af 360 grader med radius min. 3.5 m 
\item Der skal kunne tilføjes yderligere enheder efter opsætning af systemet


\subsubsection*{Vedligeholdelse}
\item Enheder skal være udskiftelige uden at det er nødvendigt at tilgå sensorer eller sprinkler
\item Sprinklere skal være let tilgængelige for personale


\subsubsection*{Generelle krav}
\item Al kommunikation sker via en seriel bus 


\subsubsection*{Enheder}
\item Enhed skal kunne operere autonomt efter denne er sat op fra Master
\item Enheder skal sende log information til Master efter hver udførte måling
\item Enheder skal sende log information til Master efter hver udført vanding

\end{enumerate}

\subsubsection*{Begrænsninger}
\begin{itemize}
\item Der udarbejdes en skaleret udgave, da vi ikke har adgang til en hel golfbane
\item Grundet tidsbegrænsninger udvikles der ikke et fuldt system 
\item Som minimum skal der produceres funktionel Master og én komplet Enhed
\item Systemet udvikles således, at der til hver sprinkler tilhører en pumpe. Denne løsning vil ikke vælges i en produktionsudgave, her vil der være én vandforsyning, med et tryk stort nok, til at tilføre samtlige sprinklere, den nødvendige mængde vand. Sprinklerne vil i denne færdige udgave styres vha. magnetventiler 
% Skal uddybes i fremtidigt atbejde ?
\end{itemize}



