\chapter{Softwaredesign}

\section{Dataprotokol (BS)}\label{header:dataprotokol}
%% SW design: Dataprotokol

Når data flyttes mellem Master og Enhed, ifm. logning, anvendes følgende dataprotokol i mellem \textit{Application}-lagene.

Den data som skal flyttes er følgende:

\begin{enumerate}
	\item Temperatur
	\item Fugtighed
	\item Bevægelsesregistrering
	\item Påbegyndt vanding
\end{enumerate}

Systemet ved ikke på forhånd hvor meget data der skal flyttes. Derfor deles det op i tre typer. Data fra sensorene, besked om bevægelse og fejlregistreringer.
Fælles for de tre er at tidsstemplet altid sendes med.

Fra standardbiblioteket for \verb'C++' vælges datatypen \verb+vector+ som container. Dette er en dynamisk array-struktur som automatisk udvides og mindskes efter behov.
Ved at anvende \verb+vector+en med \verb+string+ som undertype er det nemt at identificerer indholdene og formaterer dem som nødvendigt.

Som nævnt er der flere muligheder for ''datapakker''. I tabellerne \ref{table:SWDataprotokol-sensor}, \ref{table:SWDataprotokol-bevaegelse} og \ref{table:SWDataprotokol-error} er deres opbygninger vist.

Første streng der læses afgør hvor mange af de efterfølgende strenge som høre her til. Hvis det første der modtages er ''\verb+D+'', betyder det at der kommer information fra sensorene og der skal læses fire efterfølgende strenge.
Hvis der modtages et ''\verb+B+'' betyder det at der er registreret bevægelse, og der skal kun læses en dato efterfølgende. Hvis der modtages et ''\verb+E+'' er der en fejl, og fejlkoden fra Enheden sendes.

\begin{table}[h]
	\caption{Dataformatering ifm. sensordata}
	\centering
	\begin{tabular}{|c|c|c|c|}
		\hline 
		\textbf{Type} & \textbf{Temperatur} & \textbf{Fugtighed} & \textbf{Vanding} \\ 
		\hline 
		''\verb+D+'' & ''\verb+TTT.T+'' & ''\verb+FFF+'' & ''Tilstand'' \\ 
		\hline 
	\end{tabular} 
	\label{table:SWDataprotokol-sensor}
\end{table}

\begin{table}[h]
	\caption{Dataformatering ifm. bevægelse}
	\centering
	\begin{tabular}{|c|c|}
		\hline 
		\textbf{Type} \\ 
		\hline 
		''\verb+B+'' \\ 
		\hline 
	\end{tabular} 
	\label{table:SWDataprotokol-bevaegelse}
\end{table}

\begin{table}[h]
	\caption{Dataformatering ifm. fejl}
	\centering
	\begin{tabular}{|c|c|c|}
		\hline 
		\textbf{Type} & \textbf{Fejlkode} \\ 
		\hline 
		''\verb+E+'' & ''\verb+XXXX+'' \\ 
		\hline 
	\end{tabular} 
	\label{table:SWDataprotokol-error}
\end{table}

Et eksempel på strukturen af en datahentning er vist i tabel \ref{table:SWDataprotokol-eksempel}. Her kan man se at der siden sidste hentning har været bevægelse på hullet, der er hentet sensordata med værdierne 14.5 gader og 20\% fugtighed. Der har også været en fejl 13.

\begin{table}[h]
	\caption{Dataformatering ifm. log-information}
	\centering
	\begin{tabular}{|c|c|c|c|c|c|c|}
		\hline
		\verb+vector<string>[0]+ & \verb+[1]+ & \verb+[2]+ & \verb+[3]+ & \verb+[4]+ & \verb+[5]+ & \verb+[6]+ \\
		\hline 
		''\verb+B+'' & ''\verb+D+'' & ''\verb+014.5+'' & ''\verb+020+'' & ''\verb+Slukket+'' & ''\verb+E+'' & ''\verb+013+'' \\ 
		\hline 
	\end{tabular} 
	\label{table:SWDataprotokol-eksempel}
\end{table}

\section{Applikationsmodeller (JC BS)}
Selve designet af softwaren bygger på de følgende applikationsmodeller. Her laves der sekvens- og klassediagrammer over hver del af systemet samt klassebeskrivelser hvor funktionen for de enkelte metoder beskrives.

%% SW design: Applikationsmodeller

Selve designet af softwaren bygger på de følgende applikationsmodeller. Her laves der sekvens- og klassediagrammer over hver del af systemet samt klassebeskrivelser hvor funktionen for de enkelte metoder beskrives.

\subsection{Master}
Applikationsmodeller for Master.

\subsection{Enhed}
Applikationsmodeller for Enhed.

% Klassebeskrivelser
\subsection{Klassebeskrivelser}
Her følger klassebeskrivelser for de udledte klasser fra applikationsmodellerne.

% Master
\subsubsection{Master (JC)}
%% SW design: klassebeskrivelse SPI API

\begin{figure}[htbp] \centering
{\includegraphics[scale=1.5]{filer/design/Klassediagrammer/SPI_API}}
\caption{Klasse SPI\_api}
\label{fig:SPI API klassediagram}
\end{figure} 

{\centering
\textbf{SPI\_api}\par
}
\textbf{Ansvar:} At være et lag imellem \textit{applications}-laget og \textit{device driver}-laget ifm. SPI kommunikation. \

\verb+int activate( int unit ) const +\\
\textbf{Parametre:} Modtager en integer på Enhed som skal aktiveres. \\
\textbf{Returværdi:} 0 ved succes ellers negativ i overenstemmelse med fejl-listen. \\
\textbf{Beskrivelse:} Metoden skal aktivere Enheden \verb+unit+ over SPI netværket ved brug af kommandoen \verb+"AC"+.\\

\verb+int deactivate( int unit ) const+ \\
\textbf{Parametre:} Modtager en integer på Enhed som skal deaktiveres.\\
\textbf{Returværdi:} 0 ved succes ellers negativ i overenstemmelse med fejl-listen. \\
\textbf{Beskrivelse:} Metoden skal deaktivere Enheden \verb+unit+ over SPI netværket ved brug af kommandoen \verb+"DC"+.\\

\verb+int verify( int unit ) const+ \\
\textbf{Parametre:}  Modtager en integer på Enhed som skal verificeres.\\
\textbf{Returværdi:} 0 ved succes ellers negativ i overenstemmelse med fejl-listen.   \\
\textbf{Beskrivelse:} Metoden skal verificere om Enheden \verb+unit+ er tilkoblet SPI netværket ved at sende kommandoen \verb+"VRC"+. Når \verb+'R'+ sendes, modtages samtidigt enhedens nummer. Hvis det modtagende nummer er identisk med \verb+unit+ er verificeringen succesfuld.\\

\verb+int config( int unit, float temp, float humi ) const +\\
\textbf{Parametre:} Modtager en integer på Enhed som skal konfigureres. Derudover modtager den to floats med parametrene som skal skrives til Enhed. \\
\textbf{Returværdi:}  0 ved succes ellers negativ i overenstemmelse med fejl-listen.   \\
\textbf{Beskrivelse:} Metoden skal sende konfigurations-parametrene i \verb+float+ til Enheden \verb+unit+ over SPI netværket med brug af kommandoen \verb+"PTTT.THHHC"+. Se Kommunikationsprotokollen, \ref{header:kommunikationsprotokol}, for detaljer.\\

\verb+int getLog( vector<string> &, int * units, int size )+ \\
\textbf{Parametre:}  Modtager en reference til en \verb+vector+ af typen \verb+string+ som loggen skal gemmes i. Pointer til array af Enheder som skal logges samt integer med antallet af Enheder i arrayet. \\
\textbf{Returværdi:}  0 ved succes ellers negativ i overenstemmelse med fejl-listen.   \\
\textbf{Beskrivelse:} Metoden skal hente log fra Enheder i arrayet \verb+units+ på SPI netværket med kommandoen \verb+LRR..+ (Se Kommunikationsprotokollen, \ref{header:kommunikationsprotokol}, for detaljer) og gemme dem i \verb+vector+ i henhold til dataprotokollen, \ref{header:dataprotokol}. \\

%% SW design: klassebeskrivelse UI

\begin{figure}[htbp] \centering
{\includegraphics[scale=1.5]{filer/design/Klassediagrammer/sw_UI}}
\caption{Klasse UI}
\label{fig:UI klassediagram}
\end{figure} 

{\centering
\textbf{UI}\par
}
\textbf{Ansvar:} Håndterer alt ved den grafiske brugerflade. \

\textbf{Attributter:}
\begin{itemize}
	\item \verb+QStackedWidget winStack_+ Objekt til at holde alle vinduer i
	\item \verb+winMain winMainObj_+ Hovedmenu vindue
	\item \verb+winStatus winStatusObj_+ Vis status vindue
	\item \verb+winOnOff winOnOffObj_+ Aktiver / Deaktiver vindue
	\item \verb+winAddRemove winAddRemoveObj_+ Tilføj / fjern vindue
	\item \verb+winConfig winConfigObj_+ Konfigurer vindue
	\item \verb+winConfigPar winConfigParObj_+ Konfigurer parametre vindue
	\item \verb+winAddRemovePar winAddRemoveParObj_+ Tilføj Enheds parametre vindue
	\item \verb+winLog winLogObj_+ Vis Log vindue
	\item \verb+cStatus * cStatusPtr_+ Pointer til associeret objekt
	\item \verb+cOnOff * cOnOffPtr_+ Pointer til associeret objekt
	\item \verb+cAddRemove * cAddRemovePtr_+ Pointer til associeret objekt
	\item \verb+cConfig * cConfigPtr_+ Pointer til associeret objekt
	\item \verb+cLogView * cLogViewPtr_+ Pointer til associeret objekt
\end{itemize}

\verb+UI( )+\\
\textbf{Parametre:} Ingen \\
\textbf{Returværdi:} Ingen \\
\textbf{Beskrivelse:} Tilføjer alle \verb+win+-objekterne til stakken \verb+winStack_+. Sætter det aktive vindue til hovedmenuen \verb+winMainObj_+. Indstiller størrelsen på vinduer i stakken til at fylde hele skærmen på Devkit8000, 480x272 og sætter størrelsen på \verb+UI+ til at have samme størrelse. Til sidst opsættes den centrale widget i \verb+QMainWindow+ til stakken \verb+winStack_+.\\

\verb+int setCStatus( cStatus & )+\\
\textbf{Parametre:} Reference til associeret objekt \\
\textbf{Returværdi:} 0 ved succes ellers negativ i overenstemmelse med fejl-listen. \\
\textbf{Beskrivelse:} Sætter medlemspointer til associeret controller.\\

\verb+int setCOnOff( cOnOff & )+\\
\textbf{Parametre:} Reference til associeret objekt \\
\textbf{Returværdi:} 0 ved succes ellers negativ i overenstemmelse med fejl-listen. \\
\textbf{Beskrivelse:} Sætter medlemspointer til associeret controller.\\

\verb+int setCAddRemove( cAddRemove & )+\\
\textbf{Parametre:} Reference til associeret objekt \\
\textbf{Returværdi:} 0 ved succes ellers negativ i overenstemmelse med fejl-listen. \\
\textbf{Beskrivelse:} Sætter medlemspointer til associeret controller.\\

\verb+int setCConfig( cConfig & )+\\
\textbf{Parametre:} Reference til associeret objekt \\
\textbf{Returværdi:} 0 ved succes ellers negativ i overenstemmelse med fejl-listen. \\
\textbf{Beskrivelse:} Sætter medlemspointer til associeret controller.\\

\verb+int setCLogView( cLogView & )+\\
\textbf{Parametre:} Reference til associeret objekt \\
\textbf{Returværdi:} 0 ved succes ellers negativ i overenstemmelse med fejl-listen. \\
\textbf{Beskrivelse:} Sætter medlemspointer til associeret controller.\\

\verb+cStatus * getCStatus( ) const+\\
\textbf{Parametre:} Ingen \\
\textbf{Returværdi:} 0 ved succes ellers negativ i overenstemmelse med fejl-listen. \\
\textbf{Beskrivelse:} Returnerer pointer til associeret objekt.\\

\verb+cOnOff * getCOnOff( ) const+\\
\textbf{Parametre:} Ingen \\
\textbf{Returværdi:} 0 ved succes ellers negativ i overenstemmelse med fejl-listen. \\
\textbf{Beskrivelse:} Returnerer pointer til associeret objekt.\\

\verb+cAddRemove * getCAddRemove( ) const+\\
\textbf{Parametre:} Ingen \\
\textbf{Returværdi:} 0 ved succes ellers negativ i overenstemmelse med fejl-listen. \\
\textbf{Beskrivelse:} Returnerer pointer til associeret objekt.\\

\verb+cConfig * getCConfig( ) const+\\
\textbf{Parametre:} Ingen \\
\textbf{Returværdi:} 0 ved succes ellers negativ i overenstemmelse med fejl-listen. \\
\textbf{Beskrivelse:} Returnerer pointer til associeret objekt.\\

\verb+cLogView * getCLogView( ) const+\\
\textbf{Parametre:} Ingen \\
\textbf{Returværdi:} 0 ved succes ellers negativ i overenstemmelse med fejl-listen. \\
\textbf{Beskrivelse:} Returnerer pointer til associeret objekt.\\

\verb+winOnOff * getWinOnOff( )+\\
\textbf{Parametre:} Ingen \\
\textbf{Returværdi:} 0 ved succes ellers negativ i overenstemmelse med fejl-listen. \\
\textbf{Beskrivelse:} Returnerer pointer til associeret objekt.\\

\verb+winLog * getWinLog( )+\\
\textbf{Parametre:} Ingen \\
\textbf{Returværdi:} 0 ved succes ellers negativ i overenstemmelse med fejl-listen. \\
\textbf{Beskrivelse:} Returnerer pointer til associeret objekt.\\

\verb+winAddRemove * getWinAddRemove( )+\\
\textbf{Parametre:} Ingen \\
\textbf{Returværdi:} 0 ved succes ellers negativ i overenstemmelse med fejl-listen. \\
\textbf{Beskrivelse:} Returnerer pointer til associeret objekt.\\

\verb+winAddRemovePar * getWinAddRemovePar( )+\\
\textbf{Parametre:} Ingen \\
\textbf{Returværdi:} 0 ved succes ellers negativ i overenstemmelse med fejl-listen. \\
\textbf{Beskrivelse:} Returnerer pointer til associeret objekt.\\

\verb+winConfig * getWinConfig( )+\\
\textbf{Parametre:} Ingen \\
\textbf{Returværdi:} 0 ved succes ellers negativ i overenstemmelse med fejl-listen. \\
\textbf{Beskrivelse:} Returnerer pointer til associeret objekt.\\

\verb+winConfigPar * getWinConfigPar( )+\\
\textbf{Parametre:} Ingen \\
\textbf{Returværdi:} 0 ved succes ellers negativ i overenstemmelse med fejl-listen. \\
\textbf{Beskrivelse:} Returnerer pointer til associeret objekt.\\

\verb+winStatus * getWinStatus( )+\\
\textbf{Parametre:} Ingen \\
\textbf{Returværdi:} 0 ved succes ellers negativ i overenstemmelse med fejl-listen. \\
\textbf{Beskrivelse:} Returnerer pointer til associeret objekt.\\

\verb+QStackedWidget * getStack( )+\\
\textbf{Parametre:} Ingen \\
\textbf{Returværdi:} 0 ved succes ellers negativ i overenstemmelse med fejl-listen. \\
\textbf{Beskrivelse:} Returnerer pointer til winStack-objektet.\\

\verb+int showMain( )+\\
\textbf{Parametre:} Ingen \\
\textbf{Returværdi:} 0 ved succes ellers negativ i overenstemmelse med fejl-listen. \\
\textbf{Beskrivelse:} Sætter aktive vindue i \verb+winStack_+-objektet til \verb+winMainObj_+.\\

\verb+int showStatus( )+\\
\textbf{Parametre:} Ingen \\
\textbf{Returværdi:} 0 ved succes ellers negativ i overenstemmelse med fejl-listen. \\
\textbf{Beskrivelse:} Sætter aktive vindue i \verb+winStack_+-objektet til \verb+winStatusObj_+.\\

\verb+int showOnOff( )+\\
\textbf{Parametre:} Ingen \\
\textbf{Returværdi:} 0 ved succes ellers negativ i overenstemmelse med fejl-listen. \\
\textbf{Beskrivelse:} Sætter aktive vindue i \verb+winStack_+-objektet til \verb+winOnOffObj_+.\\

\verb+int showAddRemove( )+\\
\textbf{Parametre:} Ingen \\
\textbf{Returværdi:} 0 ved succes ellers negativ i overenstemmelse med fejl-listen. \\
\textbf{Beskrivelse:} Sætter aktive vindue i \verb+winStack_+-objektet til \verb+winAddRemoveObj_+.\\

\verb+int showConfig( )+\\
\textbf{Parametre:} Ingen \\
\textbf{Returværdi:} 0 ved succes ellers negativ i overenstemmelse med fejl-listen. \\
\textbf{Beskrivelse:} Sætter aktive vindue i \verb+winStack_+-objektet til \verb+winConfigObj_+.\\

\verb+int showParam( )+\\
\textbf{Parametre:} Ingen \\
\textbf{Returværdi:} 0 ved succes ellers negativ i overenstemmelse med fejl-listen. \\
\textbf{Beskrivelse:} Sætter aktive vindue i \verb+winStack_+-objektet til \verb+winConfigParObj_+.\\

\verb+int showAddRemovePar( )+\\
\textbf{Parametre:} Ingen \\
\textbf{Returværdi:} 0 ved succes ellers negativ i overenstemmelse med fejl-listen. \\
\textbf{Beskrivelse:} Sætter aktive vindue i \verb+winStack_+-objektet til \verb+winAddRemoveParObj_+.\\

\verb+int showLog( )+\\
\textbf{Parametre:} Ingen \\
\textbf{Returværdi:} 0 ved succes ellers negativ i overenstemmelse med fejl-listen. \\
\textbf{Beskrivelse:} Sætter aktive vindue i \verb+winStack_+-objektet til \verb+winLogObj_+.\\


%% SW design: klassebeskrivelse devkit Controllers
\newpage

\begin{figure}[htbp] \centering
{\includegraphics[scale=1.5]{filer/design/Klassediagrammer/sw_addRemove}}
\caption{klassediagram addRemove}
\label{fig:addRemove klassediagram}
\end{figure} 

{\centering
\textbf{addRemove}\par
}
\textbf{Ansvar:} at styre forløbet i UC1: Tilføj / fjern enhed. \

int menuAddRemove( ) const \\
\textbf{Parametre:} Modtager ingen parametre \\
\textbf{Returværdi:} 0 ved succes ellers negativ i overenstemmelse med fejl-listen \\
\textbf{Beskrivelse:} metoden skal styre forløbet i UC1: tilføj fjern.\\

\begin{figure}[htbp] \centering
{\includegraphics[scale=1.5]{filer/design/Klassediagrammer/sw_config}}
\caption{klassediagram config}
\label{fig:config klassediagram}
\end{figure} 

{\centering
\textbf{Config}\par
}
\textbf{Ansvar:} at styre forløbet i UC2: Konfig. \

int menuConfig( ) const \\
\textbf{Parametre:} Modtager ingen parametre \\
\textbf{Returværdi:} 0 ved succes ellers negativ i overenstemmelse med fejl-listen \\
\textbf{Beskrivelse:} metoden skal styre forløbet i UC1: tilføj fjern.\\

\begin{figure}[htbp] \centering
{\includegraphics[scale=1.5]{filer/design/Klassediagrammer/sw_onOff}}
\caption{klassediagram onOff}
\label{fig:onOff klassediagram}
\end{figure} 

\newpage

{\centering
\textbf{onOff}\par
}
\textbf{Ansvar:} at styre forløbet i UC3: Aktiver / deajtuver. \

int menuOnOff( ) const \\
\textbf{Parametre:} Modtager ingen parametre \\
\textbf{Returværdi:} 0 ved succes ellers negativ i overenstemmelse med fejl-listen \\
\textbf{Beskrivelse:} metoden skal styre forløbet i UC1: tilføj fjern.\\

\begin{figure}[htbp] \centering
{\includegraphics[scale=1.5]{filer/design/Klassediagrammer/sw_loadData}}
\caption{klassediagram loadData}
\label{fig:loadData klassediagram}
\end{figure} 

{\centering
\textbf{loadData}\par
}
\textbf{Ansvar:} at styre forløbet i UC4: Databehandling . \

int menuloadData( ) \\
\textbf{Parametre:} Modtager ingen parametre \\
\textbf{Returværdi:} 0 ved succes ellers negativ i overenstemmelse med fejl-listen \\
\textbf{Beskrivelse:} metoden skal styre forløbet i UC1: tilføj fjern.\\

\begin{figure}[htbp] \centering
{\includegraphics[scale=1.5]{filer/design/Klassediagrammer/sw_status}}
\caption{klassediagram status}
\label{fig:status klassediagram}
\end{figure} 

\newpage

{\centering
\textbf{status}\par
}
\textbf{Ansvar:} at styre forløbet i UC5: Tjek status. \

int menuStatus( ) const \\
\textbf{Parametre:} Modtager ingen parametre \\
\textbf{Returværdi:} 0 ved succes ellers negativ i overenstemmelse med fejl-listen \\
\textbf{Beskrivelse:} metoden skal styre forløbet i UC1: tilføj fjern.\\

\begin{figure}[htbp] \centering
{\includegraphics[scale=1.5]{filer/design/Klassediagrammer/sw_logView}}
\caption{klassediagram logView}
\label{fig:logView klassediagram}
\end{figure}

{\centering
\textbf{logView}\par
}
\textbf{Ansvar:} at styre forløbet i UC6: Udskriv log. \

int menuLog( ) const \\
\textbf{Parametre:} Modtager ingen parametre \\
\textbf{Returværdi:} 0 ved succes ellers negativ i overenstemmelse med fejl-listen \\
\textbf{Beskrivelse:} metoden skal styre forløbet i UC1: tilføj fjern.\\


%% SW design: klassebeskrivelse devkit domain klasser
\newpage

\begin{figure}[htbp] \centering
{\includegraphics[scale=1.5]{filer/design/Klassediagrammer/sw_unitDB}}
\caption{klassediagram unitDB}
\label{fig:unitDB klassediagram}
\end{figure} 

{\centering
\textbf{unitDB}\par
}
\textbf{Ansvar:} at styre forløbet i UC1: Tilføj / fjern enhed. \

int getUnits( \& int array, \& int size ) \\
\textbf{Parametre:}  \\
\textbf{Returværdi:} 0 ved succes ellers negativ i overenstemmelse med fejl-listen \\
\textbf{Beskrivelse:} \\

int saveUnit( int adresse, int nr ) \\
\textbf{Parametre:}  \\
\textbf{Returværdi:} 0 ved succes ellers negativ i overenstemmelse med fejl-listen \\
\textbf{Beskrivelse:} .\\

\begin{figure}[htbp] \centering
{\includegraphics[scale=1.5]{filer/design/Klassediagrammer/sw_log}}
\caption{klassediagram log}
\label{fig:log klassediagram}
\end{figure} 

{\centering
\textbf{log}\par
}
\textbf{Ansvar:} Gemme information loggen til senere brug. \

int saveLog( vector<string> ) \\
\textbf{Parametre:} Modtager en vector af typen string. \\
\textbf{Returværdi:} 0 ved succes ellers negativ i overenstemmelse med fejl-listen \\
\textbf{Beskrivelse:} Modtager log fra enhed og skriver den over i den samlede log og gemmer den i latest \\

int getLog( \& vector<string> )  \\
\textbf{Parametre:} Modtager en adresse til en vector af typen string. \\
\textbf{Returværdi:} 0 ved succes ellers negativ i overenstemmelse med fejl-listen \\
\textbf{Beskrivelse:} .\\

int getLatest( \& vector<string> ) \\
\textbf{Parametre:} Modtager en adresse til en vector af typen string.  \\
\textbf{Returværdi:} 0 ved succes ellers negativ i overenstemmelse med fejl-listen \\
\textbf{Beskrivelse:} .\\




% Enhed
\subsubsection{Enhed (BS)}
Her følger klassebeskrivelser til Enhed. 
Bemærk at alt kode til PSoC er skrevet i \verb+C+, hvorfor det ikke er muligt at lave klasser. Beskrivelserne tager dog udgangspunkt i \verb-C++-, men skal fortolkes iht. beskrivelsen i \textit{UML-Light}\footnote{T-133 UML-Light af Finn Overgaard Hansen. Benyttet i forbindelse med faget I1OPRG (Objektorienteret PRoGrammering) på 1. semester.} 

%% SW design: PSoC Controllers

\begin{figure}[htbp] \centering
{\includegraphics[scale=1.3]{filer/design/Klassediagrammer/sw_psoc_addRemove}}
\caption{Klasse addRemove}
\label{fig:sw_psoc_class_addremove}
\end{figure} 

{\centering
\textbf{addRemove}\par
}
\textbf{Ansvar:} Kontrollerer hændelsesforløbet ifm. usecase 1. \

\verb+int verify( )+\\
\textbf{Parametre:} Ingen \\
\textbf{Returværdi:} 0 ved succes ellers negativ i overenstemmelse med fejl-listen. \\
\textbf{Beskrivelse:} Returnerer kun 0. Bruges til at verificerer kommunikation mellem Master og Enhed.\\

\begin{figure}[htbp] \centering
{\includegraphics[scale=1.3]{filer/design/Klassediagrammer/sw_psoc_config}}
\caption{Klasse config}
\label{fig:sw_psoc_class_config}
\end{figure} 

{\centering
\textbf{config}\par
}
\textbf{Ansvar:} Kontrollerer hændelsesforløbet ifm. usecase 2. \

\verb+int config( float * temp const, float * humidity const )+ \\
\textbf{Parametre:} To pointere til hhv. temperatur og fugtighedsgrænser \\
\textbf{Returværdi:} 0 ved succes ellers negativ i overenstemmelse med fejl-listen. \\
\textbf{Beskrivelse:} Skal gemme parametre i et objekt af typen parameters ved at kalde metoderne \verb+setTemp()+ og \verb+setHumi()+.\\

\begin{figure}[htbp] \centering
{\includegraphics[scale=1.3]{filer/design/Klassediagrammer/sw_psoc_onOff}}
\caption{Klasse onOff}
\label{fig:sw_psoc_class_onOff}
\end{figure} 

{\centering
\textbf{onOff}\par
}
\textbf{Ansvar:} Kontrollerer hændelsesforløbet ifm. usecase 3. \

\verb+int turnOnOff( unsigned char const )+ \\
\textbf{Parametre:} En \verb+bool+ som angiver on = true, off = false \\
\textbf{Returværdi:} 0 ved succes ellers negativ i overenstemmelse med fejl-listen. \\
\textbf{Beskrivelse:} Skal sætte flaget \verb+active_+ i \verb+parameters+-objektet ud fra den modtagende parameter. Gyldige værdier er 0 og 1.\\

\begin{figure}[htbp] \centering
{\includegraphics[scale=1.3]{filer/design/Klassediagrammer/sw_psoc_loadData}}
\caption{Klasse loadData}
\label{fig:sw_psoc_class_loadData}
\end{figure} 

{\centering
\textbf{loadData}\par
}
\textbf{Ansvar:} Kontrollerer hændelsesforløbet ifm. usecase 4. \

\verb+int getBuffer( char * buf, unsigned int len )+ \\
\textbf{Parametre:} Pointer til at skrive adressen til bufferen i og en længde. \\
\textbf{Returværdi:} 0 ved succes ellers negativ i overenstemmelse med fejl-listen. \\
\textbf{Beskrivelse:} Skal hente data fra \verb+buffer+-objektet og gemme det i parametrene \verb+buf+ og \verb+len+. \\

\verb+int movementDetekt( )+ \\
\textbf{Parametre:} Ingen. \\
\textbf{Returværdi:} 0 ved succes ellers negativ i overenstemmelse med fejl-listen. \\
\textbf{Beskrivelse:} Skal deaktiverer vanding ved at sætte \verb+active_+-flaget i \verb+parameters+-objektet til 0. Skal også starte en timer med udløb på 30 minutter. Skal sætte flaget \verb+movement_+ til 1, så der næste gang der gemmes data, registreres bevægelse. \\

\verb+int logDataTimeout( )+ \\
\textbf{Parametre:} Ingen. \\
\textbf{Returværdi:} 0 ved succes ellers negativ i overenstemmelse med fejl-listen. \\
\textbf{Beskrivelse:} Skal aflæses data fra \verb+sensorPackage+ og gemme disse i \verb+buffer+-objektet. Afhængig af \verb+movement_+-flaget, tilføjes passende \verb+char+ efter målt data i hht. dataprotokollen, og flaget sættes til 0. \\

\verb+int waterTimeout( )+ \\
\textbf{Parametre:} Ingen. \\
\textbf{Returværdi:} 0 ved succes ellers negativ i overenstemmelse med fejl-listen. \\
\textbf{Beskrivelse:} Skal aktivere muligheden for vanding ved at sætte \verb+active_+-flaget i \verb+parameters+-objektet til 1. \\

%% SW design: PSoC Domain

\begin{figure}[htbp] \centering
{\includegraphics[scale=1.3]{filer/design/Klassediagrammer/sw_psoc_parameters}}
\caption{Klasse parameters}
\label{fig:sw_psoc_class_parameters}
\end{figure} 

{\centering
\textbf{parameters}\par
}
\textbf{Ansvar:} Gemme information om grænseværdier mv. til det autonome vandingssystem. \

\textbf{Attributter:}
\begin{itemize}
	\item \verb+float temperature_+ Øvre temperaturgrænse
	\item \verb+float humidity_+ Nedre fugtighedsgrænse
	\item \verb+unsigned char active_+ Flag for om vanding er aktiv eller ikke
\end{itemize}

\verb+void init( parameters * const this ) +\\
\textbf{Parametre:} Pointer til aktuelt objekt. \\
\textbf{Returværdi:} Ingen. \\
\textbf{Beskrivelse:} Initialiserer medlemsattributterne: \verb+active_+: 1, \verb+humidity_+: 10 og \verb+temperature_+: 30. \\

\verb+int setTemp( parameters * const this, const float temp ) +\\
\textbf{Parametre:} Temperatur. \\
\textbf{Returværdi:} 0 ved succes ellers negativ i overenstemmelse med fejl-listen. \\
\textbf{Beskrivelse:} Gemmer modtagne data i medlem \verb+temperature_+. \\

\verb+int getTemp( parameters * const this, float * temp )+ \\
\textbf{Parametre:} Pointer til at gemme data i. \\
\textbf{Returværdi:} 0 ved succes ellers negativ i overenstemmelse med fejl-listen. \\
\textbf{Beskrivelse:} Returnerer medlem \verb+temperature_+ i modtagne pointer. \\

\verb+int setHumi( parameters * const this, const float humi )+ \\
\textbf{Parametre:} Humidity. \\
\textbf{Returværdi:} 0 ved succes ellers negativ i overenstemmelse med fejl-listen. \\
\textbf{Beskrivelse:} Gemmer modtagne data i medlem \verb+humidity_+. \\

\verb+int getHumi( parameters * const this, float * humi )+ \\
\textbf{Parametre:} Pointer til at gemme data i. \\
\textbf{Returværdi:} 0 ved succes ellers negativ i overenstemmelse med fejl-listen. \\
\textbf{Beskrivelse:} Returnerer medlem \verb+humidity_+ i modtagne pointer. \\

\verb+int setActive( parameters * const this, const unsigned char )+ \\
\textbf{Parametre:} 1 = aktiv, 0 = inaktiv. \\
\textbf{Returværdi:} 0 ved succes ellers negativ i overenstemmelse med fejl-listen. \\
\textbf{Beskrivelse:} Gemmer modtagne data i medlem \verb+active_+. \\

\verb+int getActive( parameters * const this, unsigned char * )+ \\
\textbf{Parametre:} Pointer til at gemme data i. \\
\textbf{Returværdi:} 0 ved succes ellers negativ i overenstemmelse med fejl-listen. \\
\textbf{Beskrivelse:} Returnerer medlem \verb+active_+ i modtagne pointer. \\


\begin{figure}[htbp] \centering
{\includegraphics[scale=1.3]{filer/design/Klassediagrammer/sw_psoc_buffer}}
\caption{Klasse buffer}
\label{fig:sw_psoc_class_buffer}
\end{figure} 

{\centering
\textbf{buffer}\par
}
\textbf{Ansvar:} Holde data fra sensorerne indtil de udlæses af Master. Gemmer også fejlbeskeder fra systemet. \

\textbf{Attributter:}
\begin{itemize}
	\item \verb+char buffer_[]+ Array til at holde 1 datalæsning og 10 fejl.
	\item \verb+unsigned int len_+ Længden på arrayet
	\item \verb+unsigned int cursor_+ Index på næste frie plads i array
	\item \verb+unsigned int dataIndex_+ Index på placering af data i array, hvis der tidligere har været gemt data
	\item \verb+unsigned char dataWritten_+ Flag til at indikerer tidligere skrevet data
\end{itemize}

\verb+void buffer( buffer * )+ \\
\textbf{Parametre:} Pointer til aktuelt objekt. \\
\textbf{Returværdi:} Ingen. \\
\textbf{Beskrivelse:} r \verb+char+ array \verb+buffer_+ med plads til én datamåling og 10 fejl, iht. dataprotokol, \ref{header:dataprotokol}, til 0. Initialiserer også medlemmer til 0 på nær \verb+len_+ som initialiseres til arrayets længde \verb+BUFFER_LENGTH+.\\

\verb+int saveData( buffer * const, const char * buf, const unsigned int len )+ \\
\textbf{Parametre:} Pointer til aktuelt objekt og pointer til buffer med len data. \\
\textbf{Returværdi:} 0 ved succes ellers negativ i overenstemmelse med fejl-listen. \\
\textbf{Beskrivelse:} Gemmer modtaget data i array, hvis der er plads til det. Hvis det er en data-logning, kontrolleres der om der ligger en ældre måling i bufferen, som i så fald overskrives. Markøren flyttes iht. hvor meget der skrives ind. \\

\verb+int getData( buffer * const, char ** buf, unsigned int * len )+ \\
\textbf{Parametre:} Pointer til aktuelt objekt og pointer til char array til data. \\
\textbf{Returværdi:} 0 ved succes ellers negativ i overenstemmelse med fejl-listen. \\
\textbf{Beskrivelse:} Returnerer pointeren til array i \verb+buf+ og \verb+cursor_+ i \verb+len+. Nulstiller derefter \verb+cursor_+ og \verb+dataWritten_+.\\
%% SW design: PSoC Boundary

\begin{figure}[htbp] \centering
{\includegraphics[scale=1.3]{filer/design/Klassediagrammer/sw_psoc_sensorPackage}}
\caption{Klasse sensorPackage}
\label{fig:sw_psoc_class_sensorPackage}
\end{figure} 

{\centering
\textbf{sensorPackage}\par
}
\textbf{Ansvar:} Holder styr på tilkoblede sensorer. \

\verb+sensorPackage( ) +\\
\textbf{Parametre:} Ingen \\
\textbf{Returværdi:} Ingen \\
\textbf{Beskrivelse:} Skal oprette sensor- og sprinklerobjekter og udføre den nødvendige opsætning af disse. \\

\verb+int getData( float * temp, float * humi )+ \\
\textbf{Parametre:} Pointers til at gemme aflæste temperatur og fugtighed i. \\
\textbf{Returværdi:} 0 ved succes ellers negativ i overenstemmelse med fejl-listen. \\
\textbf{Beskrivelse:} Aflæse data fra temperatur- og fugtighedssensor og returner disse i de modtagende referencer. \\

\verb+int water( unsigned char const )+ \\
\textbf{Parametre:} 1 = tænd sprinkler, 0 = sluk sprinkler \\
\textbf{Returværdi:} 0 ved succes ellers negativ i overenstemmelse med fejl-listen. \\
\textbf{Beskrivelse:} Tænd eller sluk tilkoblede sprinkler ud fra modtagende parameter. \\


\begin{figure}[htbp] \centering
{\includegraphics[scale=1.3]{filer/design/Klassediagrammer/sw_psoc_tempSensor}}
\caption{Klasse tempSensor}
\label{fig:sw_psoc_class_tempSensor}
\end{figure} 

{\centering
\textbf{tempSensor}\par
}
\textbf{Ansvar:} Håndterer kommunikation med temperatursensor. \

\verb+tempSensor( )+ \\
\textbf{Parametre:} Ingen. \\
\textbf{Returværdi:} Ingen. \\
\textbf{Beskrivelse:} Initialisere nødvendige indstillinger for at kunne kommunikerer med sensoren. \\

\verb+int getValue( float * ) const+ \\
\textbf{Parametre:} Pointer til at gemme temperatur i. \\
\textbf{Returværdi:} 0 ved succes ellers negativ i overenstemmelse med fejl-listen. \\
\textbf{Beskrivelse:} Returnerer aflæst temperatur i reference. \\


\begin{figure}[htbp] \centering
{\includegraphics[scale=1.3]{filer/design/Klassediagrammer/sw_psoc_humiSensor}}
\caption{Klasse humiSensor}
\label{fig:sw_psoc_class_humiSensor}
\end{figure} 

{\centering
\textbf{humiSensor}\par
}
\textbf{Ansvar:} Håndterer kommunikation med fugtighedssensor. \

\verb+humiSensor( )+ \\
\textbf{Parametre:} Ingen. \\
\textbf{Returværdi:} Ingen. \\
\textbf{Beskrivelse:} Initialisere nødvendige indstillinger for at kunne kommunikerer med sensoren. \\

\verb+int getValue( float * ) const+ \\
\textbf{Parametre:} Pointer til at gemme fugtighed i. \\
\textbf{Returværdi:} 0 ved succes ellers negativ i overenstemmelse med fejl-listen. \\
\textbf{Beskrivelse:} Returnerer aflæst fugtighed i reference. \\

\begin{figure}[htbp] \centering
{\includegraphics[scale=1.3]{filer/design/Klassediagrammer/sw_psoc_sprinkler}}
\caption{Klasse sprinkler}
\label{fig:sw_psoc_class_sprinkler}
\end{figure} 

{\centering
\textbf{sprinkler}\par
}
\textbf{Ansvar:} At aktiverer og deaktiverer tilkoblede sprinkler \

\verb+sprinkler( )+ \\
\textbf{Parametre:} Ingen. \\
\textbf{Returværdi:} Ingen. \\
\textbf{Beskrivelse:} Skal initialisere nødvendige dele for at kunne bruge GPIO.\\

\verb+int setValue( unsigned char const )+ \\
\textbf{Parametre:} 1 = tænd sprinkler, 0 = sluk sprinkler. \\
\textbf{Returværdi:} 0 ved succes ellers negativ i overenstemmelse med fejl-listen. \\
\textbf{Beskrivelse:} Skal aktiverer eller deaktiverer sprinkler ved hjælp af GPIO.\\

\clearpage

% Statiske klassediagrammer
\section{Statisk klassediagram}
Efter sekvensdiagrammerne, klassediagrammet og klassebeskrivelser er designfasen så langt at der nu kan sammensættes et statisk klassediagram som viser det samlede software design. 

\subsection{Master (JC)}
%% SW klassediagram static

\begin{figure}[!htbp] \centering
{\includegraphics[scale=0.7]{filer/design/sw_class_devkit_static}}
\caption{Statisk klassediagram for Master (Devit8000)}
\label{fig:class_static_dev}
\end{figure}

Da Vi har arbejdet i Qt frameworket til at oprette og styre vores user interface på Devkit8000 så har vi gjort brug af mange af deres klasser. F.eks anvendes der ofte en QString istedet for en std::string. Der er design billeder der viser hvordan vores UI ser ud når det bliver deployed på devkittet. Design filerne er dem i UI klassen som starter med win som ses på figur \ref{fig:class_static_dev}. Disse UI billeder har nogle knapper som kan sende nogle signaler til vores funktioner, et såkaldt event system. Event systemet er det som bringer os rundt i UI'en ved at kalde nogle bestemte funktioner når de pågældende knapper bliver trykket på osv.


\clearpage

\subsection{Enhed (BS)}
%% SW klassediagram static

\begin{figure}[!htbp] \centering
{\includegraphics[scale=0.7]{filer/implementering/sw_class_psoc_static}}
\caption{Statisk klassediagram for Enhed (PSoC)}
\label{fig:class_psoc_static_dev}
\end{figure}




\clearpage

