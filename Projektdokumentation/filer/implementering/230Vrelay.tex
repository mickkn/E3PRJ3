% 230V Relæ
Ingeniørhøjskolen ved Aarhus Universitet tillader kun at arbejde med spændinger op til 48 V uanset tidligere baggrund\footnote{Poul er ikke uddannet elektriker, men har lavet el til husbehov. Relæet kan sammenlignes med en forlængerledning med en afbryder, hvilket er lovligt at lave privat}. Der er dog givet særskilt tilladelse til at bygge 230 V relæet, hvis dette indbygges i en lukket kasse som beskrevet i design delen. Relæet er godkendt af Torben Lund Jensen fra værkstedet på IHA.

Relæet er implementeret som beskrevet i designdelen. Figur \ref{lab:Relay_inside} viser relæet som det er implementeret. Veroboard er benyttet som sokkel for relæet. Installationsledning (0,75 kvadrat) er benyttet til 230 V forbindelsen.
Den lyseblå leder er nul, den sorte leder er fasen  og jord/beskyttelseslederen er
den gul/grønne. Denne går direkte fra indgang (apparatstik) til udgang (230 V stikkontakt).
  


\begin{figure}[htb]
\centering
{\includegraphics[width=0.60\textwidth]{filer/implementering/relay_inside}}
\caption{Relæet uden påmonteret låg}
\label{lab:Relay_inside}
\end{figure}

Figur \ref{lab:Relay_connection} viser relæet fra begge ender. Venstre side af figuren viser indgangen, som består af ét 230 V apparatstik og 2 bananstik. Apparatstikket tilsluttes en stikkontakt som er tændt. Bananstikkene tilsluttes tilslutningsprintet. Højre side viser 230 V stikkontakten som tændes/slukkes af relæet. Vandpumpen tilsluttes denne stikkontakt.

\begin{figure}[htb]
\centering
{\includegraphics[width=0.80\textwidth]{filer/implementering/relay_connection}}
\caption{Relæet set fra begge ender.}
\label{lab:Relay_connection}
\end{figure}
