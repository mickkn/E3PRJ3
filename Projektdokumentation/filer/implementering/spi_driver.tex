% SPI Driver

Driveren til SPI består af 2 hovedmetoder, der er en skrive-metoder og en læse-metode. Disse sender og læser karaktere (chars) én ad gangen. Der er taget udgangspunkt i HAL Exercise 7\footnote{Hardware abstraktioner. Exercise 7: LDD with SPI. Øvelse med SPI Kommunikation}, men netop læse og skrive metoderne er blevet ændret for at tilpasse EasyWater8000 behov for datatransmission.

\subsubsection*{Write}

Write metoden modtager en datachar, som den lægger i tx-bufferen og skriver ud i spi\_sync. 
U8 cmd håndteres ikke for EasyWater8000, men er bibeholdt så den evt. kan bruges til andre projekter.

\begin{lstlisting}[language=C]
int psoc4_spi_write_reg8(struct spi_device *spi, u8 cmd, char data)
{
	struct spi_transfer t[1];	// Number of packages
	struct spi_message m;

	/* Check for valid spi device */
	if(!spi)
		return -ENODEV;

	/* Init Message */
	memset(&t, 0, sizeof(t)); 
	spi_message_init(&m);
	m.spi = spi;

	/* Configure tx/rx buffers */
	t[0].tx_buf = &data;		// Write data to tx-buffer
	t[0].rx_buf = NULL;		// Do nothing with rx-buffer
	t[0].len = 1;			
	spi_message_add_tail(&t[0], &m);

	/* Transmit SPI Data (blocking) */
	spi_sync(m.spi, &m);		// Transmit SPI data

	return 0;
}
\end{lstlisting}

\subsubsection*{Read}

Read-metoden skriver altid et '\verb+R+' til target i form af en hardcoded char, derfor bruges \verb+u8 addr+ ikke i EasyWater8000 projektet. I samme transmission, læses der fra rx-bufferen på target. Det indsatte delay i transmissionen, er som sådan ikke nødvendig, da der kun læses én gang fra target, og mellem hver læsning er der minimum 300 µs, men det er beholdt for en sikkerheds skyld, da det ikke er problematisk at en læsning tager 100us mere.

Rx-bufferen skrives til data, som flyttes til value char-pointeren, til videre håndtering af Masteren.

\begin{lstlisting}[language=C]
int psoc4_spi_read_reg8(struct spi_device *spi, u8 addr, char* value)
{
	struct spi_transfer t[1];
	struct spi_message m;
	char data;		// Temp. var to hold readings
	char cmd = 'R';		// Hardcoded 'R' char

	/* Check for valid spi device */
	if(!spi)
		return -ENODEV;

	/* Init Message */
	memset(t, 0, sizeof(t));
	spi_message_init(&m);
	m.spi = spi;

	/* Configure tx/rx buffers */
	t[0].delay_usecs = 100;		// Delay for PSoC4 operation
	t[0].tx_buf = &cmd;		// Write 'R' to tx-buffer
	t[0].rx_buf = &data;		// Read data-char to data variable
	t[0].len = 1;
	spi_message_add_tail(&t[0], &m);

	/* Transmit SPI Data (blocking) */
	spi_sync(m.spi, &m);			// Transmit SPI data

	*value = data;				// Move char to value
	return 0;
}
\end{lstlisting}