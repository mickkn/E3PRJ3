\subsection{Strømforsyning}

For at frigøre sig fra laboratoriets strømforsyningerne udarbejdes egen strømforsyning. Vha. en 230Vac/12Vdc transformerer og egnede spændingens-regulatorer designes en strømforsyning til netop dette projekt. Masteren (DevKit8000) har sin egen strømforsyning og denne ændres ikke. Enheden (PSoC4) skal forsynes med 5V dc via USB. Relæstyringen skal forsynes med 5V dc. PIR sensoren skal forsynes med 5-20 V dc. Temperatur/fugt sensoren skal forsynes med 3,3V dc. 

230Vac/12Vdc transformeren er fra en gammel computer. Denne kan give 2 amperre og er forsynet med et DC stik (12V) og en stikprop (230Vac). Denne transformer bruges som den er. 

For at regulere fra de 12 V til 5 V benyttes en spændingsregulator. LM7805 er beregnet netop til dette formål. Ud fra databladet \ref{lab:LM7805} ses forbindelsen af LM7805'ern. Databladet foreskriver også at regulatoren kan give et output på max 1A ved korrekt køling. 

\begin{figure}[H] \centering
{\includegraphics[width=\textwidth]{filer/design/Billeder/LM7805_DATASHEET}}
\caption{Forbindelse af LM7805 - Side 18/24 LM7805.pdf}
\label{lab:LM7805}
\raggedright
\end{figure}

Spændingsregulatoren LM7805 er opbygget i mutilsim, sammen med 12V dc forsyningen. Kredløbet er simuleret og dette viser at outputtet er de ønskede 5 V.

\begin{figure}[H] \centering
{\includegraphics[width=\textwidth]{filer/design/Billeder/LM7805_SIMULATION}}
\caption{LM7805 Simulering}
\label{lab:LM7805_SIMULERING}
\raggedright
\end{figure}


For at opnå 3,3V benyttes en justerbar spændingsregulator LM317. Denne regulator kan levere 1,2V-33V 3A. Det er fugt,-temperatursensoren der skal forsynes med 3,3V/180uA, det er LM317'erns opgave. 

LM317 er slået op iPad appen "Electronic Toolbox", vha. denne gives hvordan opsætningen af denne skal være for at opnå 3,3V 
 
\begin{figure}[H] \centering
{\includegraphics[width=\textwidth]{filer/design/Billeder/LM317}}
\caption{Kredsløb for LM137}
\label{lab:LM317}
\raggedright
\end{figure}

Figur \ref{lab:LM317_calc} viser hvorledes appen beregner en værdi for modstanden R2, når Vout er sat til 3,3V og modstanden R1 anbefales til 240 ohm. I følge appens beregning skal R2 være 393,6 ohm.

\begin{figure}[H] \centering
{\includegraphics[width=\textwidth]{filer/design/Billeder/LM317_calc}}
\caption{Kredsløb for LM137}
\label{lab:LM317_calc}
\raggedright
\end{figure}

Figur \ref{lab:LM317_SIMULERING} viser simuleringen af LM317 spændingsregulatoren. Outputtet er 3,3V som forventet. Der simuleres med en modstand R2 på 390 ohm og ikke 393,6 ohm.

\begin{figure}[H] \centering
{\includegraphics[width=\textwidth]{filer/design/Billeder/LM317_SIMULATION}}
\caption{Simulering i Multisim ud fra Electronic Toolbox appens værdier}
\label{lab:LM317_SIMULERING}
\raggedright
\end{figure}


\subsection{Tilslutningsprint}

Tilslutningsprintet, se figur \ref{lab:PSU_connections}, sørger for at sammenkoble Enheden med de ønskede sensorer, samt at forsyne Enhed og sensorer med ønsket forsyning.

\begin{figure}[H] \centering
{\includegraphics[width=\textwidth]{filer/design/Billeder/tilslutningsprint}}
\caption{Kredsløb over samlet spændingsforsyning (3,3V og 5V) og tilslutninger til PSoC4}
\label{lab:PSU_connections}
\raggedright
\end{figure}

