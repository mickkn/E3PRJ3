%% SW design: Dataprotokol

Når data flyttes mellem Master og Enhed, ifm. logning, anvendes følgende dataprotokol i mellem \textit{Application}-lagene.

Den data som skal flyttes er følgende:

\begin{enumerate}
	\item Temperatur
	\item Fugtighed
	\item Bevægelsesregistrering
	\item Påbegyndt vanding
\end{enumerate}

Systemet ved ikke på forhånd hvor meget data der skal flyttes. Derfor deles det op i tre typer. Data fra sensorene, besked om bevægelse og fejlregistreringer.
Fælles for de tre er at tidsstemplet altid sendes med.

Fra standardbiblioteket for \verb'C++' vælges datatypen \verb+vector+ som container. Dette er en dynamisk array-struktur som automatisk udvides og mindskes efter behov.
Ved at anvende \verb+vector+en med \verb+string+ som undertype er det nemt at identificerer indholdene og formaterer dem som nødvendigt.

Som nævnt er der flere muligheder for ''datapakker''. I tabellerne \ref{table:SWDataprotokol-sensor}, \ref{table:SWDataprotokol-bevaegelse} og \ref{table:SWDataprotokol-error} er deres opbygninger vist.

Første streng der læses afgør hvor mange af de efterfølgende strenge som høre her til. Hvis det første der modtages er ''\verb+D+'', betyder det at der kommer information fra sensorene og der skal læses fire efterfølgende strenge.
Hvis der modtages et ''\verb+B+'' betyder det at der er registreret bevægelse, og der skal kun læses en dato efterfølgende. Hvis der modtages et ''\verb+E+'' er der en fejl, og fejlkoden fra Enheden sendes.

\begin{table}[h]
	\caption{Dataformatering ifm. sensordata}
	\centering
	\begin{tabular}{|c|c|c|c|}
		\hline 
		\textbf{Type} & \textbf{Temperatur} & \textbf{Fugtighed} & \textbf{Vanding} \\ 
		\hline 
		''\verb+D+'' & ''\verb+TTT.T+'' & ''\verb+FFF+'' & ''Tilstand'' \\ 
		\hline 
	\end{tabular} 
	\label{table:SWDataprotokol-sensor}
\end{table}

\begin{table}[h]
	\caption{Dataformatering ifm. bevægelse}
	\centering
	\begin{tabular}{|c|c|}
		\hline 
		\textbf{Type} \\ 
		\hline 
		''\verb+B+'' \\ 
		\hline 
	\end{tabular} 
	\label{table:SWDataprotokol-bevaegelse}
\end{table}

\begin{table}[h]
	\caption{Dataformatering ifm. fejl}
	\centering
	\begin{tabular}{|c|c|c|}
		\hline 
		\textbf{Type} & \textbf{Fejlkode} \\ 
		\hline 
		''\verb+E+'' & ''\verb+XXXX+'' \\ 
		\hline 
	\end{tabular} 
	\label{table:SWDataprotokol-error}
\end{table}

Et eksempel på strukturen af en datahentning er vist i tabel \ref{table:SWDataprotokol-eksempel}. Her kan man se at der siden sidste hentning har været bevægelse på hullet, der er hentet sensordata med værdierne 14.5 gader og 20\% fugtighed. Der har også været en fejl 13.

\begin{table}[h]
	\caption{Dataformatering ifm. log-information}
	\centering
	\begin{tabular}{|c|c|c|c|c|c|c|}
		\hline
		\verb+vector<string>[0]+ & \verb+[1]+ & \verb+[2]+ & \verb+[3]+ & \verb+[4]+ & \verb+[5]+ & \verb+[6]+ \\
		\hline 
		''\verb+B+'' & ''\verb+D+'' & ''\verb+014.5+'' & ''\verb+020+'' & ''\verb+Slukket+'' & ''\verb+E+'' & ''\verb+013+'' \\ 
		\hline 
	\end{tabular} 
	\label{table:SWDataprotokol-eksempel}
\end{table}