%% SW design: Dataprotokol

\subsubsection{Log}

Når data flyttes data mellem Master og Enhed, ifm. logning, anvendes følgende dataprotokol i mellem \textit{Application}-lagene.

Den data som skal flyttes er følgende:

\begin{enumerate}
	\item Temperatur
	\item Fugtighed
	\item Bevægelsesregistrering
	\item Påbegyndt vanding
\end{enumerate}

Systemet ved ikke på forhånd hvor meget data der skal flyttes. Derfor deles det op i to typer. Data fra sensorene samt besked om bevægelse og vanding, og fejlregistreringer.

Fra standardbiblioteket for \verb'C++' vælges datatypen \verb+vector+ som container. Dette er en dynamisk array-struktur som automatisk udvides og mindskes efter behov.
Ved at anvende \verb+vector+en med \verb+string+ som undertype er det nemt at identificerer indholdene og formaterer dem som nødvendigt. På Enheden som implementeres på en PSoC er standardbiblioteket til \verb-C++- ikke tilrådighed. Derfor implementeres bufferen på denne som et almindeligt statisk allokeret \verb+char+-array. Formateringen vist her under gælder for begge platforme.

Som nævnt er der flere muligheder for ''datapakker''. I tabellerne \ref{table:SWDataprotokol-sensor} og \ref{table:SWDataprotokol-error} er deres opbygninger vist.

Første streng der læses afgør hvor mange af de efterfølgende strenge som høre her til. Hvis det første der modtages er ''\verb+D+'', betyder det at der kommer information fra sensorene og der skal læses fire efterfølgende strenge.
Hvis der modtages et ''\verb+E+'' er der en fejl og én streng med fejlkoden fra enheden læses.

\begin{table}[h]
	\caption{Dataformatering ifm. sensordata}
	\centering
	\begin{tabular}{|c|c|c|c|c|}
		\hline 
		\textbf{Type} & \textbf{Temperatur} & \textbf{Fugtighed} & \textbf{Vanding} & \textbf{Bevægelse}\\ 
		\hline 
		''\verb+D+'' & ''\verb+TTT.T+'' & ''\verb+FFF+'' & ''\verb+0+'' / ''\verb+1+'' & ''\verb+0+'' / ''\verb+1+'' \\ 
		\hline 
	\end{tabular} 
	\label{table:SWDataprotokol-sensor}
\end{table}

\begin{table}[h]
	\caption{Dataformatering ifm. fejl}
	\centering
	\begin{tabular}{|c|c|}
		\hline 
		\textbf{Type} & \textbf{Fejlkode} \\ 
		\hline 
		''\verb+E+'' & ''\verb+XXX+'' \\ 
		\hline 
	\end{tabular} 
	\label{table:SWDataprotokol-error}
\end{table}

Et eksempel på strukturen af en datahentning er vist i tabel \ref{table:SWDataprotokol-eksempel}. Her kan man se, at der siden sidste udlæsning har været hentet sensordata med værdierne 14.5 grader og 20\% fugtighed, der har ikke været tændt for sprinkleren men der har været bevægelse på hullet, \verb+D014.502001+ og der har også været en fejl 13, \verb+E013+. Bemærk at rækkefølgen ikke er bestemt på forhånd. Det er altså muligt at der kun ligger fejl på en Enhed, hvorfor der vil blive læst et antal fejl ud, men ingen datapakker.

\begin{table}[h]
	\caption{Dataformatering ifm. log-information}
	\centering
	\begin{tabular}{|c|c|}
		\hline
		\verb+vector<string>[0]+ & \verb+[1]+ \\
		\hline 
		''\verb+D014.502001+'' & ''\verb+E013+'' \\ 
		\hline 
	\end{tabular} 
	\label{table:SWDataprotokol-eksempel}
\end{table}