% SPI API

SPI\_api er en klasse til at styre SPI-kommunikationen. Den består af en række metoder beskrevet i afsnit \ref{fig:SPI API klassediagram}. I de følgende sektioner beskrives i pseudo-kode, hvordan metoderne skal implementeres. Alle metoderne afsluttes med at sende et 'C' til Enheden. Det sørger for at TX-bufferen til SPI nulstilles, så den er klar til næste opgave.

\subsubsection*{int activate(int unit) const}

Metoden skal sørge for at åbne kernefilen, skrive til den og lukke den igen.

\begin{lstlisting}[language=C]
Open file
Write 'A' to file
Write 'C' to file
Close file
Return 0
\end{lstlisting} 
\subsubsection*{int deactivate(int unit) const}

Metoden skal sørge for at åbne kernefilen, skrive til den og lukke den igen.

\begin{lstlisting}[language=C]
Open file
Write 'D' to file
Write 'C' to file
Close file
Return 0
\end{lstlisting} 
\subsubsection*{int verify(int unit) const} 

Metoden skal sørge for at åbne kernefilen, skrive til den, læse fra den og lukke den igen.
Den skal også sammenligne parameternummeret og det udlæste enhedsnummer fra Enheden og returnere 0 hvis der er et match og en fejlkode ved mismatch.

\begin{lstlisting}[language=C]
Open file
Write 'V' to file
Read unitNo from file
Parse unitNo to int
Write 'C' to file
Close file
Compare unitNo with unit
Return 0 if match
Return error if mismatch
\end{lstlisting} 
\subsubsection*{int config(int unit, float temp, float humi)} 

Metoden skal sørge for at åbne kernefilen, skrive til den og lukke den igen.
Den håndterer 2 floats, som skal konverteres til chars for at kunne sendes via SPI.

\begin{lstlisting}[language=C]
Open file
Write 'P' to file
Write temp to file
Write humi to file
Write 'C' to file
Close file
Return 0
\end{lstlisting} 
\subsubsection*{int getLog(vector<string> \&data, int * units, int size)}

Metoden sørger for at åbne kernefilen, skrive til den, læse fra den og lukke den igen.
Den udlæser et buffer array fra Enheden som pakkes ned i en vector til data parameteret.

\begin{lstlisting}[language=C]
Open file
Write 'L' to file
Read buffersize from file
Read buffer buffersize times
Build datastring if 'D' are present
Build errorstring(s) if 'E' are present
Build vector from data and errorstrings
Move vector to data
Close file
Return 0
\end{lstlisting} 