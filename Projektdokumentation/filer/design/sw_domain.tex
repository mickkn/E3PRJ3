%% SW design: klassebeskrivelse devkit domain klasser
\newpage

\begin{figure}[htbp] \centering
{\includegraphics[scale=1.5]{filer/design/Klassediagrammer/sw_unitDB}}
\caption{klassediagram unitDB}
\label{fig:unitDB klassediagram}
\end{figure} 

{\centering
\textbf{unitDB}\par
}
\textbf{Ansvar:} at holde styr på hvor mange enheder der er og deres adresser. \

int getUnits( \& int array, \& int size ) \\
\textbf{Parametre:}  Modtager en adresse på et array af ints og en tilhørende referance til at skrive size i.\\
\textbf{Returværdi:} 0 ved succes ellers negativ i overenstemmelse med fejl-listen \\
\textbf{Beskrivelse:} Metoden skriver sit adresse array over i det array som modtages. Derudover skriver den størrelsen på arrayet over i size så modtageren ved hvor stort arrayet er. (hvor mange golfhuller der er)\\

%int saveUnit( int adresse, int nr = size_ ) \\
\textbf{Parametre:} Modtager en adresse af typen int og et index nr af typen int. \\
\textbf{Returværdi:} 0 ved succes ellers negativ i overenstemmelse med fejl-listen \\
%\textbf{Beskrivelse:}  Metoden gemmer adressen som modtages på plads nr. Hvis der ikke modtages et index nr, så lægges adressen i array[size] og lægger 1 til size_. \\

\begin{figure}[htbp] \centering
{\includegraphics[scale=1.5]{filer/design/Klassediagrammer/sw_log}}
\caption{klassediagram log}
\label{fig:log klassediagram}
\end{figure} 

{\centering
\textbf{log}\par
}
\textbf{Ansvar:} Gemme information loggen til senere brug. \

int saveLog( vector<string> ) \\
\textbf{Parametre:} Modtager en vector af typen string. \\
\textbf{Returværdi:} 0 ved succes ellers negativ i overenstemmelse med fejl-listen \\
\textbf{Beskrivelse:} Modtager log fra enhed og skriver den over i den samlede log og gemmer den i latest. Loggen skal gemmes i en txt fil som der læses fra under startup. \\

int getLog( \& vector<string> )  \\
\textbf{Parametre:} Modtager en adresse til en vector af typen string. \\
\textbf{Returværdi:} 0 ved succes ellers negativ i overenstemmelse med fejl-listen \\
\textbf{Beskrivelse:} Metoden skriver klassens medlems data log over i den adresse som den har modtaget som parameter.\\

int getLatest( \& vector<string> ) \\
\textbf{Parametre:} Modtager en adresse til en vector af typen string.  \\
\textbf{Returværdi:} 0 ved succes ellers negativ i overenstemmelse med fejl-listen \\
\textbf{Beskrivelse:} Metoden skriver klassens medlems data latest over i den adresse som den har modtaget som parameter.\\


