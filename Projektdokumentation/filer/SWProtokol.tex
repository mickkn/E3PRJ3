%% SW arkitektur: Kommunikationsprotokol

En del data skal flyttes mellem Master og de tilkoblede Enheder. Her følger beskrivelsen af hvordan data pakkes mellem de to dele. Den elektriske protokol som anvendes er SPI.

Opsætningen er som følger:
\textit{Hastighed
SPI mode
Antal bits}

Ud fra UC-beskrivelserne er der identificeret følgende scenarier hvor der sendes data mellem Master og Enhed.

\begin{enumerate}
	\item Master kontrollerer om Enhed er koblet til systemet (UC1)
	\item Master sender parametre til Enhed (UC2)
	\item Enhed aktiveres eller deaktiveres af Master (UC3)
	\item Master beder om data fra Enhed (UC4)
	\item Enhed sender data til Master (UC4)
\end{enumerate}

Fælles for alle kommandoer er start- og stopkarakteren. Her vælges karakterene vist i tabel \ref{table:SWProtokol-char}.

\begin{table}[h]
	\caption{Start- og stopbytes for SPI-kommunikationen}
	\centering
	\begin{tabular}{|c|c|c|}
		\hline 
		& \textbf{ASCII} & \textbf{Hex} \\ 
		\hline 
		\textbf{STX} & 'S' / 's' & 0x53 / 0x73 \\ 
		\hline 
		\textbf{ETX} & '\textbackslash r' & 0x0D \\ 
		\hline 
	\end{tabular} 
	\label{table:SWProtokol-char}
\end{table}

Dataen til og fra enhederne pakkes i nogle frames som beskrevet i tabel \ref{table:SWProtokol-frames}. Først sendes STX, der efter kommer længden af det efterfølgende data. Her efter kommer kommandoen og evt. data i tilfælde af UC2 og UC4, og der afsluttes med ETX.

\begin{table}[h]
	\caption{Data formatering for SPI-kommunikation}
	\centering
	\begin{tabular}{|l|c|c|c|c|c|}
		\hline 
		\textbf{Byte} & 0 & 1 & 2 & 3..X & X + 1 \\ 
		\hline 
		\textbf{Indhold} & STX & <Længde> & <Kommando> & <Data> & ETX \\ 
		\hline 
	\end{tabular} 
	\label{table:SWProtokol-frames}
\end{table}

\subsubsection*{Blokken <Længde>}
Længde-byten bestemmer hvor meget data der sendes med den pågældende kommando.
Hvis der ikke sendes andet end selve kommandoen er den 0. Bemærk at dette skal fortolkes som et heltal og ikke en karakter! Der skal sendes \textit{4} ikke '\textit{4}'.

\subsubsection*{Blokken <Kommando>}
Alle kommandoer er én byte lang og kommandoerne i tabel \ref{tabel:SWProtokol-kommandoer} er de tilgængelige kommandoer. Hvis ikke kommandoen genkendes er der intet svar.

\begin{table}[h]
\caption{Kommandoer for SPI-kommunikation}
\centering
\begin{tabular}{|c|c|l|c|}
\hline 
\textbf{ASCII} & \textbf{HEX} & \textbf{Funktion} & \textbf{Afsender}\\ 
\hline 
'A' / 'a' & 0x41 / 0x61 & Aktiver Enhed\\ 
\hline 
'D' / 'd' & 0x44 / 0x64 & Deaktiver Enhed\\ 
\hline 
'V' / 'v' & 0x56 / 0x76 & Verificer Enhed i systemet\\ 
\hline 
'P' / 'p' & 0x50 / 0x70 & Parametre sendes til Enhed\\ 
\hline
'L' / 'l' & 0x4c / 0x6c & Forespørg logdata fra Enhed\\ 
\hline
'M' / 'm' & 0x4d / 0x6d & Returner antal af bytes i buffer\\ 
\hline
\end{tabular}
\label{tabel:SWProtokol-kommandoer}
\end{table} 

\subsubsection*{Blokken <Data>}
Data-blokken bruges til at sende data mellem enhederne. Her følger beskrivelsen for hvordan denne formateres. Bemærk den kun bruges til kommandoen \textit{'P'} (UC2) og \textit{'L'} (UC4).

\textbf{Send parametre}

Her skal sendes følgende parametre til Enheden.

\begin{enumerate}
	\item Nedre fugtighedsgrænse
	\item Øvre temperaturgrænse
\end{enumerate}

Disse består af tocifrede tal i hhv. procent og grader Celsius. 
Parametrene skal sendes i ovenstående rækkefølge og resulterer altså i en kommando som vist i tabel \ref{table:SWProtokol-para}.

\begin{table}[h]
	\caption{Data-formatering for parameter-kommando}
	\centering
	\begin{tabular}{|l|c|c|c|c|c|c|}
		\hline 
		\textbf{Byte} & 0 & 1 & 2 & 3..4 & 5..6 & 7 \\ 
		\hline 
		\textbf{Indhold} & STX & 4 & P & <Fugt> & <Temp> & ETX \\ 
		\hline 
	\end{tabular} 
	\label{table:SWProtokol-para}
\end{table}

\textbf{Send log}

Når masteren skal udhente afventende data i Enheden ved denne ikke på forhånd hvor meget data der skal modtages. Da Masteren kun kan starte kommunikation i SPI startes med at sende en 'L' kommando og efterfølgende læses én byte som returneres af Enheden. Her i står antallet af bytes som skal læses fra Enheden. Masteren påbegynder der efter en række læsninger svarende til antallet af bytes.

Et forløb kunne være som følgende.
Master sender strengen ''\textit{S0L\textbackslash r}'' og læser her efter bytes indtil ETX modtages.
Svaret fra Enheden kan være på to former. Den ene er retursvar for logning af KP-data. Denne er vist i tabel \ref{table:SWProtokol-log-kp}. Det kan også være en bevægelse der er registreret. Denne formateres som vist i tabel  \ref{table:SWProtokol-log-bev}.

\begin{table}[h]
	\caption{Data-formatering for KP-returværdi på Hent log kommando}
	\centering
	\begin{tabular}{|l|c|c|c|c|c|}
		\hline 
		\textbf{Byte} & 0 & 1 & 2..3 & 4..5 & 6 \\ 
		\hline 
		\textbf{Indhold} & <KP-nr> & <Tid> & <Fugt> & <Temp> & <Sprinkler> \\ 
		\hline 
	\end{tabular} 
	\label{table:SWProtokol-log-kp}
\end{table}

\begin{table}[h]
	\caption{Data-formatering for Bevægelses-returværdi på Hent log kommando}
	\centering
	\begin{tabular}{|l|c|c|}
		\hline 
		\textbf{Byte} & 0 & 1\\ 
		\hline 
		\textbf{Indhold} & <Bevægelse> & <Tid> \\ 
		\hline 
	\end{tabular} 
	\label{table:SWProtokol-log-bev}
\end{table}