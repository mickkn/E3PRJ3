%% SW arkitektur: Deployment View

Deployment view skal illustrere hvor hvilke lag af software ligger på vores platforme. Devkit8000 kører linux og har derfor flere software lag end PSoC'en.
 
\vspace{15 mm}

\begin{figure}[htbp] \centering
{\includegraphics[scale=0.7]{filer/systemarkitektur/Deployment_model}}
\caption{Deployment model illustrere de forskellige software og hardware(grønne) lag}
\label{fig:Deployment Model}
\end{figure}

\vspace{5 mm}

\subsubsection{Devkit8000}
\textit{Applications} laget består af alt den software som har med brugeren at gøre, dvs. UI og tilhørende controllers. Applicationslaget tager imod input fra brugeren og reagere på det enten ved at kalde nogle af sine egne controllers eller sende kommandoer til hardware APIen. Laget skal desuden få data fra nedenstående lag til at fremstå overskueligt over for brugeren.

\clearpage

\textit{Hardware API} laget består af almindelige klasser som gør brug af file systemets kommandoer som f.eks open og close.

\textit{File System}

\textit{Device Driver} laget består af den software som håndtere alt hardware input og output. Hardware proxys inklusiv.

\textit{HW connection (grønne bokse)} laget viser hvilke hardware in/out der er til devkittet

\subsubsection{PSoC}

\textit{Applications} laget håndtere den indsamlede data som den får fra sensorene igennem API'en. Denne data sammensættes ifølge protokollen og sendes til API'en som får det sendt til devkittet. Laget skal også håndtere data fra devkittet til at konfigurere de parametre der styre automatiseringen af vandingen som applications laget også håndtere.

\textit{API} laget består af den software som håndtere hardwaren. dvs den tager imod input og får formateret det til noget læseligt til applications laget. Derudover står den for at få sendt de informationer applicationslaget ber om på en effektiv og sikker måde.

\textit{HW connection (grønne bokse)} laget viser hvilke hardware in/out der er til PSoC'en