For at danne overblik over software-udviklingen inden det egentlige design laves bruges N+1 modellen.
Denne beskriver fire faser som tager hånd om de overordnede ting inden for software, alt sammen med use-cases som den røde tråd.
De fire fase er:
\begin{enumerate}
	\item Logical View
	\item Deployment View
	\item Implementation View
	\item Data View
\end{enumerate}

Ud over disse punkter tænkes der også en overordnet fejl-håndtering ind i projektet samt skitser for den grafiske brugerflade. Disse punkter er beskrevet i detaljer her efter.

\subsection{Logical View (JSA)}
%% SW arkitektur: Logical View

Logical View skal danne et overblik over hvilke softwarepakker der befinder sig på vores platforme. Blokkene inde i de respektive pakker kan sammen med domænemodellen hjælpe med at give et overblik over hvilke klasser og kernemoduler der skal bruges.

\begin{figure}[htbp] \centering
{\includegraphics[scale=0.7]{filer/systemarkitektur/logical_view_devkit}}
\caption{Logical view for Devkit8000 illustrerer hvilke softwarepakker der befinder sig på Masteren }
\label{fig:Logical View Devkit8000}
\end{figure}

Figur \ref{fig:Logical View Devkit8000} illustrerer hvilke softwarepakker der ligger på Masteren. I bunden er \textit{Device drivers}-pakken som håndterer SPI kommunikationen imellem Devkit8000 og PSoC. I midten ligger \textit{Hardware API}-pakken som håndterer protokol-vedtægter ifm. kommunikationen. \textit{Application}-pakken tager sig af alt UI samt log og fejlhåndtering.

\clearpage

\newenvironment{figure1}[1][]{\begin{figure}[#1]\vspace{3.0cm}}{\vspace{1.0cm}\end{figure}}
\begin{figure1}[htbp] \centering
{\includegraphics[scale=0.7]{filer/systemarkitektur/logical_view_psoc}}
\caption{Logical view for PSoC illustrer hvilke software pakker der befinder sig på enhederne}
\label{fig:Logical View PSoC}
\end{figure1}

Figur \ref{fig:Logical View PSoC} illustrerer hvilke softwarepakker der ligger på Enhederne. \textit{API}-pakken består af den software som håndterer hardwaren, dvs. den tager imod input og får formateret det til noget brugbart for \textit{Application}-pakken. \textit{Application}-pakken håndterer den indsamlede data som den får fra sensorene igennem \textit{API}-pakken. Denne data sammensættes iht. protokollen og sendes til API pakken som får det sendt til Devkit8000. Pakken skal også håndterer data fra Devkit8000 til at konfigurerer de parametre der styrer automatiseringen af vandingen.



\subsection{Deployment View ()}
%% SW arkitektur: Deployment View

Deployment view skal illustrere hvor hvilke lag af software ligger på vores platforme. Devkit8000 kører Linux og har derfor flere softwarelag end PSoC'en.
 
\vspace{15 mm}

\begin{figure}[htbp] \centering
{\includegraphics[scale=0.7]{filer/systemarkitektur/Deployment_model}}
\caption{Deployment model illustrerer de forskellige software og hardware(grønne) lag}
\label{fig:Deployment Model}
\end{figure}

<<<<<<< HEAD
=======
\vspace{5 mm}

\subsubsection{Devkit8000}
\textit{Applications} laget består af al den software som har med brugeren at gøre, dvs. UI og tilhørende controllers. Applicationslaget tager imod input fra brugeren og reagerer på det enten ved at kalde nogle af sine egne controllers eller sende kommandoer til hardware APIen. Laget skal desuden få data fra nedenstående lag til at fremstå overskueligt over for brugeren.

>>>>>>> origin/master
\clearpage

\begin{figure}[!htbp] \centering
{\includegraphics[scale=0.7]{filer/systemarkitektur/IBD_deployment}}
\caption{IBD med software packages}
\label{fig:IBD deployment}
\end{figure}



\subsection{Implementation View ()}
%% SW arkitektur: Implementation View

Inden programmerne designes, fastlægges en struktur for kildekoden. På den måde er det nemmere for flere programmører at arbejde med delene i programmet samtidigt.

Strukturen skal være som vist i figur \ref{fig:implementationview}. Under mappen ''Kildekode'' skal hver klasse have en mappe med der til hørende filer. Ligeledes med ''Testprogrammer'' mappen, som indholder testprogrammer som verificerer funktionaliteten af de enkelte moduler.
Mappen ''Kompilerede programmer'' er til de endelige programmer.

\begin{figure}[htbp] \centering
{\includegraphics[scale=0.7]{filer/pics/SW-Implementation-View}}
\caption{Mappestruktur for software}
\label{fig:implementationview}
\end{figure}

\subsection{Data View (BS)}
%% SW arkitektur: Data View

I forbindelse med EasyWater8000s log skal der gemmes data på en nem og håndterbar måde. Det skal være muligt at gemme følgende data:

\begin{enumerate}
	\item Tidsstempel
	\item Temperatur
	\item Fugtighed
	\item Bevægelse
	\item Vanding
\end{enumerate}

Ydermere skal disse informationer gemmes for hver Enhed. Så hvis der er 18 huller med i alt 18 Enheder, skal ovenstående gemmes for alle 18 Enheder.

Når informationen skal præsenteres for brugeren skal det ske i en tabel som vist på figur \ref{fig:GUI-log-alle} i afsnit \ref{subsec:GUI}, data for en enkelt Enhed som på figur \ref{fig:GUI-log-enhed} i afsnit \ref{subsec:GUI} eller på en graf så man kan se ændringer over tid som vist på figur \ref{fig:log-graf}.

\begin{figure}[htbp] \centering
{\includegraphics[scale=0.5]{filer/pics/SW-Log-graf}}
\caption{Graf for temperatur registeret på en Enhed}
\label{fig:log-graf}
\end{figure}

Tabellen skal have en fane for hver opkoblet Enhed. Her kan man se informationer fra hver enkelt Enhed. Det bør også være muligt at se alle Enheders informationer samtidigt.

Dataen struktureres i semikolon-separerede filer (\verb+.csv+) på Master. Hver Enhed har en fil hvor alle data er samlet med udgangspunkt i strukturen vist i liste \ref{list:log-csv-struktur}.

\begin{lstlisting}[caption=Semikolon-separeret datafil til log af enheder, label={list:log-csv-struktur}]
<enheds-nr>;
<KP-nr>; <dato>; <temperatur>; <fugtighed>; <bevaeglse>; <vanding>;
<KP-nr>; <dato>; <temperatur>; <fugtighed>; <bevaeglse>; <vanding>;
...
<KP-nr>; <dato>; <temperatur>; <fugtighed>; <bevaeglse>; <vanding>;
\end{lstlisting}

Dette resulterer i en filstruktur som vist på liste \ref{list:log-fil-struktur} hvis der er koblet 18 Enheder op på Master.

\begin{lstlisting}[caption=Filstruktur for logfiler på Master, label={list:log-fil-struktur}]
<log>/
  <enheds-nr1>.csv
  <enheds-nr2>.csv
  ...
  <enheds-nr17>.csv
  <enheds-nr18>.csv
\end{lstlisting}

Hyppigheden for målingerne og logningen er beskrevet i de ikke-funktionelle krav i afsnit \ref{header:ikke-funk}.


\subsection{Fejl-håndtering (JC)}
%% SW arkitektur: Fejlhåndtering

EasyWater8000 kan håndtere fejl og disse vil blive gemt i en fejllog som bliver gemt på masteren. 

Fejlhåndteringen bliver klaret af en klasse på devkittet som håndtere at skrive det rigtige fejl ud i en txt fil. Klassen vil blive kaldt med en fejlkode hver gang fejl opstår. Klassen forstår så at skrive den rigtige fejl ind i txt filen ud fra den pågældende fejlkode den har modtaget som attribut. 

Alle funktioner vil returnere en negativ integer som repræsentere en fejlkoden som ErrorHandling klassen kan tolke på.

\begin{figure}[htbp] \centering
{\includegraphics[scale=0.7]{filer/pics/Errortxt}}
\caption{udsnit af Error log}
\label{fig:ErrorLog}
\end{figure}

Billedet ovenfor viser hvordan 2 fejl i error loggen evt. kunne se ud.

\subsection{Grafiske brugerflade-skitser (BS)}
\label{subsec:GUI}
%% SW arkitektur: GUI skitser

Inden det endelige GUI udvikles laves nogle skitser som udviklingen læner sig op af. Ud fra UC beskrivelserne findes de skærmbilleder som skal vises på Master og skitseres.

Her følger korte beskrivelser af hver skitse samt skitsen selv.

\subsubsection{Startmenu}
Figur \ref{fig:GUI-Startmenu} er den første menu Brugeren kommer til. Her kan Brugeren vælge hvilke funktioner vedkommende ønsker udført i systemet.

\begin{figure}[htbp] \centering
{\includegraphics[scale=0.5]{filer/pics/GUI/Start-menu}}
\caption{Skitse af startmenu på GUI}
\label{fig:GUI-Startmenu}
\end{figure}

\subsubsection{Vis Status}
På figur \ref{fig:GUI-aktuel-status} vises den aktuelle status af systemets Enheder. Den øverste række tal er tilkoblede Enheder. Den anden række tal er komponentpakker tilkoblet den enkelte Enhed. Brugeren kan altså bevæge sig rundt og se den seneste udlæsning af data fra de tilkoblede Enheder og deres komponentpakker.

\begin{figure}[htbp] \centering
{\includegraphics[scale=0.5]{filer/pics/GUI/Aktuel-status}}
\caption{Skitse af ''Vis status'' på GUI}
\label{fig:GUI-aktuel-status}
\end{figure}

\subsubsection{Aktiver og deaktiver enhed}
Skitsen på figur \ref{fig:GUI-aktiver-deaktiver} viser Brugerens mulighed for at aktivere og deaktivere enkelte enheder i systemet. Der præsenteres en liste af opsatte Enheder, hvor man kan markerer den ønskede Enhed og vælge en funktion for denne.

\begin{figure}[htbp] \centering
{\includegraphics[scale=0.5]{filer/pics/GUI/Aktiver-deaktiver-enheder}}
\caption{Skitse af ''Aktiver og deaktiver enhed'' på GUI}
\label{fig:GUI-aktiver-deaktiver}
\end{figure}

\subsubsection{Tilføj/fjern enheder}
De præsenterede muligheder i forbindelse med at fjerne og tilføje enheder til systemet er vist på figur \ref{fig:GUI-tilfoj-fjern}. Igen vises en liste af opsatte Enheder. Det er muligt at tilføje en ny, eller markerer en eksisterende og fjerne denne.

\begin{figure}[htbp] \centering
{\includegraphics[scale=0.5]{filer/pics/GUI/Tilfoj-fjern-enheder}}
\caption{Skitse af ''Tilføj/fjern enheder'' på GUI}
\label{fig:GUI-tilfoj-fjern}
\end{figure}

\subsubsection{Konfigurer}
Når brugeren skal konfigurere en Enhed, bruges skitsen på figur \ref{fig:GUI-konfigurer}. Her vælger Brugeren hvilken Enhed der skal konfigureres. Vinduet skifter til ''Indstil parametre''-vinduet når en enhed markeres.

\begin{figure}[htbp] \centering
{\includegraphics[scale=0.5]{filer/pics/GUI/Konfigurer-enhed}}
\caption{Skitse af ''Konfigurer'' på GUI}
\label{fig:GUI-konfigurer}
\end{figure}

\subsubsection{Indstil parametre}
Når brugeren har valgt en enhed som skal konfigureres vises skærmen på figur \ref{fig:GUI-indstil-parametre}. Her kan brugeren indtaste parametrene for enheden og gemme dem i systemet.

\begin{figure}[htbp] \centering
{\includegraphics[scale=0.5]{filer/pics/GUI/Indstil-parametre}}
\caption{Skitse af ''Indstil parametre'' på GUI}
\label{fig:GUI-indstil-parametre}
\end{figure}

\subsubsection{Log}
Loggen præsenterer de indsamlede data fra alle Enhederne. Det er muligt at se alle data på en gang, som vist på figur \ref{fig:GUI-log-alle}, hvor den øverste fanerække er numre på opsatte Enheder i systemet, eller man kan vælge de enkelte Enheder og se deres forskellige komponentpakker, som vist på figur \ref{fig:GUI-log-enhed}, hvor den anden fanerække er komponentpakker koblet op på systemet.

\begin{figure}[htbp] \centering
{\includegraphics[scale=0.5]{filer/pics/GUI/Log-alle}}
\caption{Skitse af ''Log'' for alle enheder på GUI}
\label{fig:GUI-log-alle}
\end{figure}

\begin{figure}[htbp] \centering
{\includegraphics[scale=0.5]{filer/pics/GUI/Log-enhed}}
\caption{Skitse af ''Log'' for enkelt enhed på GUI}
\label{fig:GUI-log-enhed}
\end{figure}



