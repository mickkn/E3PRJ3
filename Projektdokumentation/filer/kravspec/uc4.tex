\begin{center} \centering \label{UC4}
	\begin{longtable}{|p{5cm}|p{9cm}|}  %% Longtable = forsætter på næste side
	\hline
		\multicolumn{2}{|l|}{\textbf{UC4: Config}} \\\hline %% HUSK UCECASE NUMMER + NAVN
		\endfirsthead
		
		\multicolumn{2}{l}{...fortsat fra forrige side} \\ \hline %% Til LONGTABLE
		\multicolumn{2}{|l|}{\textbf{UC4: Config}} \\\hline %% HUSK UCECASE NUMMER + NAVN
		\endhead	
		
		\textbf{Mål}								&Konfigurer indstillinger på en enhed			\\\hline
		\textbf{Initialisering}					&Bruger åbner Config i UI		\\\hline
		\textbf{Aktører og Stakeholders}			&Bruger(Primær)					\\\hline
		\textbf{Referencer}						&Ingen							\\\hline
		\textbf{AASH}							&1								\\\hline
		\textbf{Forudsætning}					&Aktivt system					\\\hline
		\textbf{Efterfølgende tilstand}			&Ingen							\\\hline
		\textbf{Hovedforløb}					
			&\begin{enumerate}
	
				\item Bruger vælger Config i UI
				
				\item Bruger vælger den enhed brugeren ønsker at konfigurere.
				
				\item Bruger opsætter de ønskede indstillinger på den valgte enhed.
				
				\item Bruger færdiggøre indstillingen og vælger "Gem"
				
				\item Master uploader/programmere de nye indstillinger til den valgte enhed. 
	
			\end{enumerate}\\\hline
	\end{longtable}
\end{center}

%% TIPS:
%% LABEL TIL PUNKT: \label{labelnavn}
%% REFERENCE: \ref{labelnavn}