\begin{center} \centering \label{UC2} 
	\begin{longtable}{|p{5cm}|p{9cm}|}  %% Longtable = forsætter på næste side
	\hline
		\multicolumn{2}{|l|}{\textbf{UC2: Aktiver/deaktiver}} \\\hline %% HUSK USECASENUMMER + NAVN
		\endfirsthead
		
		\multicolumn{2}{l}{...fortsat fra forrige side} \\ \hline %% Til LONGTABLE
		\multicolumn{2}{|l|}{\textbf{UC2: Aktiver/deaktiver}} \\\hline %% HUSK USECASENUMMER + NAVN
		\endhead	
		
		\textbf{Mål}								&At Bruger kan aktivere og deaktivere Enheder	\\\hline
		\textbf{Initialisering}					&Bruger				\\\hline
		\textbf{Aktører og Stakeholders}			&Bruger(primær)		\newline 
												 Enhed(sekundær)		\newline 
												 Master(sekundær)	\\\hline
		\textbf{Referencer}						&Ingen				\\\hline
		\textbf{AASH}							&1					\\\hline
		\textbf{Forudsætning}					&Master er opsat og systemet kører \newline
												 Forbindelsen er intakt	\\\hline
		\textbf{Efterfølgende tilstand}			&Enhedstilstand ændres iflg. valg\\\hline
		\textbf{Hovedforløb}					
			&\begin{enumerate}
	
				\item Bruger logger ind på Master
				
				\item Bruger vælger aktiver/deaktiver i hovedmenuen
				
				\item Bruger vælger aktiver eller deaktiver efter ønske
				
				\item Bruger indtaster adresse på valgte enhed for at aktivere eller deaktivere denne	
	
			\end{enumerate}\\\hline
	\end{longtable}
\end{center}

%% TIPS:
%% LABEL TIL PUNKT: \label{labelnavn}
%% REFERENCE: \ref{labelnavn}