\begin{center} \centering
	\begin{longtable}{|p{6cm}|p{8cm}|}  %% Longtable = forsætter på næste side
	\hline
		\multicolumn{2}{|l|}{\textbf{UC2: Aktiver/deaktiver}} \\\hline %% HUSK UCECASE NUMMER + NAVN
		\endfirsthead
		
		\multicolumn{2}{l}{...fortsat fra forrige side} \\ \hline %% Til LONGTABLE
		\multicolumn{2}{|l|}{\textbf{UC2: Aktiver/deaktiver}} \\\hline %% HUSK UCECASE NUMMER + NAVN
		\endhead	
		
		\textbf{Mål}								&At bruger kan aktivere og/eller deaktivere enheder			\\\hline
		\textbf{Initialisering}					&Bruger			\\\hline
		\textbf{Aktører og Stakeholders}			&Bruger(primær), PsoC(sekundær), devkit8000(sekundær)			\\\hline
		\textbf{Referencer}						&Ingen			\\\hline
		\textbf{Antal af samtidige hændelser}	&1			\\\hline
		\textbf{Forudsætning}					&devkit8000 er opsat og systemet er idel. Seriel bus er intakt.			\\\hline
		\textbf{Efterfølgende tilstand}			&Ingen			\\\hline
		\textbf{Hovedforløb}					
			&\begin{enumerate}
	
				\item Bruger logger ind på devkit8000
				
				\item Bruger vælger aktiver/deaktiver i hovedmenuen
				
				\item Bruger vælger aktiver eller deaktiver efter ønske
				
				\item Bruger indtaste adresse på valgte enhed for at aktivere eller deaktivere denne
	
			\end{enumerate}\\\hline
	\end{longtable}
	\label{UC2} 
\end{center}

%% TIPS:
%% LABEL TIL PUNKT: \label{labelnavn}
%% REFERENCE: \ref{labelnavn}