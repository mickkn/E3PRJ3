\begin{center} \centering \label{UC2}
	\begin{longtable}{|p{5cm}|p{9cm}|}  %% Longtable = forsætter på næste side
	\hline
		\multicolumn{2}{|l|}{\textbf{UC2: Konfig}} \\\hline %% HUSK UCECASE NUMMER + NAVN
		\endfirsthead
		
		\multicolumn{2}{l}{...fortsat fra forrige side} \\ \hline %% Til LONGTABLE
		\multicolumn{2}{|l|}{\textbf{UC2: Konfig}} \\\hline %% HUSK UCECASE NUMMER + NAVN
		\endhead	
		
		\textbf{Mål}								&Ændre default parametre på en Enhed			\\\hline
		\textbf{Initialisering}					&Bruger							\\\hline
		\textbf{Aktører og Stakeholders}			&Bruger(Primær)					\\\hline
		\textbf{Referencer}						&Ingen							\\\hline
		\textbf{AASH}							&1								\\\hline
		\textbf{Forudsætning}					&Aktiv Master, samt at den pågældende Enhed er tilføjet systemet					\\\hline
		\textbf{Efterfølgende tilstand}			&Ingen							\\\hline
		\textbf{Hovedforløb}					
			&\begin{enumerate}
	
				\item Bruger vælger Konfig i hovedmenu
				
				\item \label{uc2afbryd}Bruger vælger den Enhed, der ønskes omkonfigureret.\newline
				\textbf{[Undtagelse \ref{uc2afbryd}.a]} \newline
					Bruger vælger afbryd
				
				\item Bruger ændrer parametre på den valgte Enhed som f.eks. fugtighedsgrænsen.
				
				\item \label{uc2afbryd2} Bruger vælger ''Gem''\newline
				\textbf{[Undtagelse \ref{uc2afbryd2}.a]} \newline
					Bruger vælger afbryd
				
				\item Master sender de nye indstillinger til den valgte Enhed. 
	
			\end{enumerate}\\\hline
		\textbf{Undtagelser}
			&\begin{enumerate}[label=\ref{uc2afbryd}.a]
				
				\item Bruger vælger ''afbryd''
				
					\subitem Skærmen viser hovedmenu
			\end{enumerate}
			
			\begin{enumerate}[label=\ref{uc2afbryd2}.a]
				
				\item Bruger vælger ''afbryd''
				
				\subitem Skærmen viser Konfig-menu

			\end{enumerate}														\\\hline
	\end{longtable}
\end{center}

%% TIPS:
%% LABEL TIL PUNKT: \label{labelnavn}
%% REFERENCE: \ref{labelnavn}