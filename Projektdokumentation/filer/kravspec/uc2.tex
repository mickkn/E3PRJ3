\begin{center} \centering \label{UC2}
	\begin{longtable}{|p{5cm}|p{9cm}|}  %% Longtable = forsætter på næste side
	\hline
		\multicolumn{2}{|l|}{\textbf{UC2: Konfig}} \\\hline %% HUSK UCECASE NUMMER + NAVN
		\endfirsthead
		
		\multicolumn{2}{l}{...fortsat fra forrige side} \\ \hline %% Til LONGTABLE
		\multicolumn{2}{|l|}{\textbf{UC2: Konfig}} \\\hline %% HUSK UCECASE NUMMER + NAVN
		\endhead	
		
		\textbf{Mål}								&Konfigurer indstillinger på en Enhed			\\\hline
		\textbf{Initialisering}					&Bruger							\\\hline
		\textbf{Aktører og Stakeholders}			&Bruger(Primær)					\\\hline
		\textbf{Referencer}						&Ingen							\\\hline
		\textbf{AASH}							&1								\\\hline
		\textbf{Forudsætning}					&Aktiv Master					\\\hline
		\textbf{Efterfølgende tilstand}			&Ingen							\\\hline
		\textbf{Hovedforløb}					
			&\begin{enumerate}
	
				\item Bruger vælger Konfig i hovedmenu
				
				\item Bruger vælger den Enhed brugeren ønsker at konfigurere.
				
				\item Bruger indstiller de ønskede indstillinger på den valgte Enhed.
				
				\item Bruger vælger ''Gem''
				
				\item Master programmere de nye indstillinger til den valgte Enhed. 
	
			\end{enumerate}\\\hline
	\end{longtable}
\end{center}

%% TIPS:
%% LABEL TIL PUNKT: \label{labelnavn}
%% REFERENCE: \ref{labelnavn}