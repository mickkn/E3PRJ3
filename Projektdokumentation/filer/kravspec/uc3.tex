\begin{center} \centering \label{UC3} 
	\begin{longtable}{|p{5cm}|p{9cm}|}  %% Longtable = forsætter på næste side
	\hline
		\multicolumn{2}{|l|}{\textbf{UC3: Aktiver/deaktiver}} \\\hline %% HUSK USECASENUMMER + NAVN
		\endfirsthead
		
		\multicolumn{2}{l}{...fortsat fra forrige side} \\ \hline %% Til LONGTABLE
		\multicolumn{2}{|l|}{\textbf{UC3: Aktiver/deaktiver}} \\\hline %% HUSK USECASENUMMER + NAVN
		\endhead	
		
		\textbf{Mål}								&At aktivere / deaktivere Enhed	\\\hline
		\textbf{Initialisering}					&Bruger				\\\hline
		\textbf{Aktører og Stakeholders}			&Bruger(Primær)		\\\hline
		\textbf{Referencer}						&Ingen				\\\hline
		\textbf{AASH}							&1					\\\hline
		\textbf{Efterfølgende tilstand}			&Enhedstilstand ændres til aktiv/deaktiv\\\hline
		\textbf{Hovedforløb}					
			&\begin{enumerate}
	
	
				\item Bruger vælger ''Aktiver/deaktiver'' i hovedmenuen
				
				\item Bruger markerer en Enhed ud fra en liste af opsatte Enheder
				
				\item \label{uc3aktiver} Bruger vælger ''Aktiver'' eller ''Deaktiver'' for at ændre Enheden\newline
				\textbf{[Undtagelse \ref{uc3aktiver}.a]} \newline
				Bruger vælger ''Tilbage''
				
				\item Master udskriver på skærmen, at ønsket Enhed er aktiveret eller deaktiveret		
			
				\item Bruger vælger ''Tilbage''
				
				\item Master viser hovedmenuen	
	
			\end{enumerate}\\\hline
			
		\textbf{Undtagelser}
			&\begin{enumerate}[label=\ref{uc3aktiver}.a]
				
				\item Bruger vælger ''Tilbage''
				
					\subitem Master viser hovedmenuen
			\end{enumerate}\\\hline			
			
	\end{longtable}
\end{center}

%% TIPS:
%% LABEL TIL PUNKT: \label{labelnavn}
%% REFERENCE: \ref{labelnavn}