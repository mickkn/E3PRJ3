\begin{center} \centering \label{UC3} 
	\begin{longtable}{|p{5cm}|p{9cm}|}  %% Longtable = forsætter på næste side
	\hline
		\multicolumn{2}{|l|}{\textbf{UC3: Aktiver/deaktiver}} \\\hline %% HUSK USECASENUMMER + NAVN
		\endfirsthead
		
		\multicolumn{2}{l}{...fortsat fra forrige side} \\ \hline %% Til LONGTABLE
		\multicolumn{2}{|l|}{\textbf{UC3: Aktiver/deaktiver}} \\\hline %% HUSK USECASENUMMER + NAVN
		\endhead	
		
		\textbf{Mål}								&At aktivere / deaktivere Enhed	\\\hline
		\textbf{Initialisering}					&Bruger				\\\hline
		\textbf{Aktører og Stakeholders}			&Bruger(primær)		\\\hline
		\textbf{Referencer}						&UC5: Vanding		\\\hline
		\textbf{AASH}							&1					\\\hline
		\textbf{Forudsætning}					&Master er opsat og systemet kører \newline
												 Forbindelsen er intakt	\\\hline
		\textbf{Efterfølgende tilstand}			&Enhedstilstand ændres aktiv/deaktiv\\\hline
		\textbf{Hovedforløb}					
			&\begin{enumerate}
	
				\item Bruger vælger ''aktiver/deaktiver'' i hovedmenuen
				
				\item Bruger vælger ''aktiver-enhed'' eller ''deaktiver-enhed''
				
				\item Bruger vælger adresse, ud fra liste, på valgte enhed for at aktivere eller deaktivere denne
				
				\item Master udskriver på dennes skærm, at ønsket Enhed er enten aktiveret eller deaktiveret	
				\item Bruger vælger ''afslut''
				
				\item Master viser hovedmenu
	
			\end{enumerate}\\\hline
	\end{longtable}
\end{center}

%% TIPS:
%% LABEL TIL PUNKT: \label{labelnavn}
%% REFERENCE: \ref{labelnavn}