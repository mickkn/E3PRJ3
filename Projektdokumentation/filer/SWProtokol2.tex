%% SW arkitektur: Kommunikationsprotokol

En del data skal flyttes mellem Master og de tilkoblede Enheder. Her følger beskrivelsen af hvordan kommunikationen mellem Master og Enhed foregår. Den elektriske protokol som anvendes er SPI.
Information omkring pakningen af data forefindes under Dataprotokol.

Opsætningen er som følger:

\begin{itemize}
  \item Hastighed: 1 MHz
  \item SPI mode: 0 (CPOL 0 - CPHA 0)
  \item Antal bits: 1 char pr. transmission
\end{itemize}

Hastigheden er valgt på baggrund af I3HAL Exercise 7\footnote{Hardware abstraktioner. Exercise 7: LDD with SPI. Øvelse med SPI Kommunikation}. PSoC'en kan køre 8MHz, men der er ikke behov for så høj hastighed. Stabiliteten blev forbedret væsentligt ved at vælge en lavere hastighed. 
\newline SPI mode: 0 er valgt på baggrund ad default indstillinger. 
\newline CPOL = 0 vil sige at clocken er lav når den er passiv (aktiv-høj). 
\newline CPHA = 0 vil sige at data udlæses på rising-edge. 
\newline Der transmitteres en karakter pr. transmission dvs. 8 bits.

Ud fra UC-beskrivelserne er der identificeret følgende scenarier hvor der sendes data mellem Master og Enhed.

\begin{enumerate}
	\item Master kontrollerer om Enhed er koblet til systemet (UC1)
	\item Master sender parametre til Enhed (UC2)
	\item Enhed aktiveres eller deaktiveres af Master (UC3)
	\item Master beder om data fra Enhed (UC4)
	\item Enhed sender data til Master (UC4)
\end{enumerate}


\subsubsection*{Kommandoer til SPI-kommunikationen}

\begin{table}[H]
\caption{Kommandoer for SPI-kommunikation}
\centering
\begin{tabular}{|c|c|l|c|}
\hline 
\textbf{ASCII} & \textbf{HEX} & \textbf{Funktion} \\ 
\hline 
'A' & 0x41 & Aktiver Enhed \\ 
\hline 
'D' & 0x44 & Deaktiver Enhed \\ 
\hline 
'P' & 0x50 & Parametre sendes til Enhed \\
\hline 
'V' & 0x56 & Verificer Enhed i systemet \\ 
\hline
'L' & 0x4c & Forespørg logdata fra Enhed \\ 
\hline
'C' & 0x4c & Write buffer og clearing af tx-buffer  \\
\hline
'R' & 0x4c & Læsning af data fra Enhed \\
\end{tabular}
\label{tabel:SWProtokol-kommandoer}
\end{table} 

