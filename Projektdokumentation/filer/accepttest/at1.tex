%%% Tabel opsætning + Headernavn!!
\begin{center}
\begin{longtable}{|p{0,65cm}|p{3,8cm}|p{3,8cm}|p{2,2cm}|p{2,2cm}|} % l for left, c for center, r for right 
\hline
\multicolumn{5}{|l|}{\textbf{UC1: Tilføj/fjern enhed}} \\ \hline
\multicolumn{1}{|c|}{} &
\textbf{Test} &
\textbf{Forventet \newline Resultat} &
\textbf{Resultat} &
\textbf{Godkendt/ \newline Kommentar} \\ \hline 
\endfirsthead

\multicolumn{5}{l}{...fortsat fra forrige side} \\ \hline 
\multicolumn{1}{|c|}{} &
\textbf{Test} &
\textbf{Forventet \newline Resultat} &
\textbf{Resultat} &
\textbf{Godkendt/ \newline Kommentar} \\ \hline 
\endhead

%%%% Tabel Opsætning

\textbf{1}	&Bruger vælger ''Tilføj/ fjern enhed'' i hovedmenu 
			&Menu fremkommer på skærmen 
			&N/A 
			&N/A \\\hline
			 
\textbf{2}	&Bruger vælger "Tilføj enhed" 
			&Menu for "Tilføj enhed" fremkommer på masters skærm  
			&N/A 
			&N/A \\\hline
			 
\textbf{3}	&Der udføres visuelt en test for om listen med opsatte enheder fremkommer
			&Listen vises for bruger med opsatte enheder
			&N/A 
			&N/A \\\hline
			 
\textbf{3.a}	&Der testes visuelt om indtastning er mulig
			&Felt for indtastningen er tilrådighed for bruger
			&N/A
			&N/A \\\hline 
			
\textbf{3.b}&Bruger indtaster label og adresse for enhed
			&Den indtastede information vises på skærmen 
			&N/A 
			&N/A \\\hline
			
\textbf{3a.a}&Bruger indtaster ugyldig information indtastes
			&Master giver fejlmeddelelse omkring ugyldig indtastning
			&N/A 
			&N/A \\\hline
			 
\textbf{3.c}&Bruger gemmer den nye enhed
			&Enheden tilføjes med de indtastede informationer
			&N/A 
			&N/A \\\hline
			 
\textbf{3.d}&Enheden forbindes fysisk til resten af systemet
			&Seriel forbindelse etableres og kommunikationen fungerer korrekt 
			&N/A
			&N/A \\\hline  
			
\textbf{3.e}&Der testes visuelt at Master giver accept for korrekt opsætning
			&Master giver accept at forbindelsen til enhed er funktionel
			&N/A 
			&N/A \\\hline 
			
\textbf{3e.a}&Forbindelsen mellem enhed og master afbrydes
			&Master giver besked om mistet kommunikation
			&N/A 
			&N/A \\\hline
			
\textbf{4}	&Bruger vælger "Fjern enhed" 
			&Menu for "Fjern enhed" fremkommer på skærmen 
			&N/A 
			&N/A \\\hline
			 
\textbf{5}	&Der udføres visuelt en test for om listen med opsatte enheder fremkommer 
			&Listen præsenteres på skærmen 
			&N/A 
			&N/A \\\hline
			 
\textbf{5.a}&Bruger indtaster adresse på enhed der ønskes deaktiveret 
			&Den indtastede information fremkommer på skærmen
			&N/A 
			&N/A \\\hline
			 
\textbf{5.b}&Der testes visuelt at Master giver information om at enheden er deaktiveret
			&Adressen accepteres og enheden deaktiveres
			&N/A 
			&N/A \\\hline
			
\textbf{5.c}&Der testes visuelt at master giver information omkring at enheden er slettet fra systemet
			&Al information om enhed slettes fra systemet 
			&N/A 
			&N/A \\\hline
			 
\textbf{6}	&Der testes visuelt at enheden er fjernet fra listen over opsatte enheder 
			&Den opdateret liste præsenteres på skærmen 
			&N/A 
			&N/A \\\hline
			 
\textbf{7.a}	&Brugeren vælger at returnere til hovedmenuen
			&Master returnerer til hovedmenuen
			&N/A
			&N/A \\\hline

\textbf{7.b}	&Brugeren vælger at fortsætte med at tilføje/fjerne enheder
			&Master viser menu for tiltøj/fjern enhed 
			&N/A
			&N/A \\\hline

\end{longtable}
	\label{ATUC1} 
\end{center}