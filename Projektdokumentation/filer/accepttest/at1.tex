% AT 1
\begin{longtable}{|p{5mm}|p{40mm}|p{40mm}|p{20mm}|p{25mm}|}
\hline 
\multicolumn{5}{|l|}{\textbf{UC1: Tilføj/fjern enhed}} \\ 
\hline 
& \textbf{Test} & \textbf{Forventede resultat} & \textbf{Resultat} & \textbf{Godkendt / kommentar} \\ 
\hline 
1 & Bruger vælger ''Tilføj/fjern enhed'' i hovedmenu & Menu fremkommer på skærmen & N/A & N/A \\ 
\hline 
2 & Bruger kan vælge "Tilføj enhed" eller "Fjern enhed" & Mulighederne vises på skærmen & N/A & N/A \\ 
\hline 
3 & Bruger vælger "Tilføj enhed" & Menu for "Tilføj enhed" fremkommer på skærmen & N/A & N/A \\ 
\hline 
4 & En liste af opsatte enheder præsenteres på skærmen & Listen præsenteres på skærmen & N/A & N/A \\ 
\hline 
4.a & Master beder bruger om at indtaste informationer & Felt fremkommer hvori information kan indtastes & N/A & N/A \\ 
\hline
4.b & Bruger indtaster navn og adresse på enhed & Indtastede information fremkommer på skærmen & N/A & N/A \\ 
\hline 
4.c & Master tilføjer enhed til systemet & Korrekt information lagers i hukommelse og bekræftelse vises på skærmen & N/A & N/A \\ 
\hline 
4.d & Enhed forbindes til kommunikationsnetværket & Seriel forbindelse etableres & N/A & N/A \\ 
\hline  
4.e & Master verificerer forbindelse til enhed & Master kontrollere om enhed er korrekt opsat og giver bruger information herom & N/A & N/A \\ 
\hline 
5 & Bruger vælger "Fjern enhed" & Menu for "Fjern enhed" fremkommer på skærmen & N/A & N/A \\ 
\hline 
6 & En liste af opsatte enheder præsenteres på skærmen & Listen præsenteres på skærmen & N/A & N/A \\ 
\hline 
6.a & Bruger indtaster adresse på enhed & Felt til indtastning fremkommer på skærmen & N/A & N/A \\ 
\hline 
6.b & Master deaktiver enheden & Adressen på den indtastede enhed deaktiveres & N/A & N/A \\ 
\hline
6.c & Master sletter enhed fra systemet & Al information om enhed slettes fra systemet & N/A & N/A \\ 
\hline 
7 & Master opdatere listen men opsatte enheder & Den opdateret liste præsenteres på skærmen & N/A & N/A \\ 
\hline 
8 & Brugeren kan returnere til hovedmenuen eller opsætte ny enhed & Master giver mulighed for at returnere eller fortsætte med at tilføje/fjerne enheder & N/A & N/A \\ 
\hline  
\end{longtable}