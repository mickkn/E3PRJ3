%%% Tabel opsætning + Headernavn!!
\begin{center}
\begin{longtable}{|p{0,65cm}|p{3,8cm}|p{3,8cm}|p{2,2cm}|p{2,2cm}|} % l for left, c for center, r for right 
\hline
\multicolumn{5}{|l|}{\textbf{UC1: Tilføj/fjern enhed}} \\ \hline
\multicolumn{1}{|c|}{} &
\textbf{Test} &
\textbf{Forventet \newline Resultat} &
\textbf{Resultat} &
\textbf{Godkendt/ \newline Kommentar} \\ \hline 
\endfirsthead

\multicolumn{5}{l}{...fortsat fra forrige side} \\ \hline 
\multicolumn{1}{|c|}{} &
\textbf{Test} &
\textbf{Forventet \newline Resultat} &
\textbf{Resultat} &
\textbf{Godkendt/ \newline Kommentar} \\ \hline 
\endhead

%%%% Tabel Opsætning

\textbf{1}	&Bruger vælger ''Tilføj/ fjern enhed'' i hovedmenu 
			&Visuel: ''Tilføj/fjern''-menu åbner på skærmen 
			&N/A 
			&N/A \\\hline
			 
\textbf{2}	&Bruger vælger "Tilføj enhed" 
			&Visuel: ''Tilføj enhed''-menu åbner på skærmen  
			&N/A 
			&N/A \\\hline
			 
\textbf{3}	&En liste af opsatte enheder præsenteres på skærmen
			&Visuel: Listen vises for bruger med evt. opsatte Enheder
			&N/A 
			&N/A \\\hline
			 
\textbf{3a}	&Master beder bruger om at indtaste informationer
			&Visuel: Det er muligt at indtaste informationer
			&N/A
			&N/A \\\hline 
			
\textbf{3b}&Bruger indtaster hul-nr. <1> og adresse på Enhed <1>
			&Visuel: Den indtastede information vises på skærmen 
			&N/A 
			&N/A \\\hline
			 
\textbf{3b.a}&Bruger indtaster navn: <tom> og adresse: <99>
			&Master giver fejlmeddelelse omkring ugyldig indtastning
			&N/A 
			&N/A \\\hline
						 
\textbf{3c}&Master tilføjer Enhed til systemet med default parametre
			&Visuel: Enheden tilføjes med de indtastede informationer
			&N/A 
			&N/A \\\hline
			 
\textbf{3d}&Enheden forbindes til kommunikationsnetværket
			&Visuel: Stik er koblet korrekt i iht. tabel \ref{table:enhed_forbindelse}
			&N/A
			&N/A \\\hline  
			
\textbf{3e}&Master verificere forbindelse til Enhed
			&Visuel: Master viser godkendt Enhed
			&N/A 
			&N/A \\\hline 
			
\textbf{3e.a}&Forbindelsen mellem Enhed og Master afbrydes
			&Visuel: Master giver fejlbesked og mulighed for at forsøge igen eller gå tilbage
			&N/A 
			&N/A \\\hline
						
\textbf{4}	&Bruger vælger ''Fjern enhed'' 
			&Visuel: ''Fjern enhed''-menu åbner på skærmenen 
			&N/A 
			&N/A \\\hline
			 
\textbf{5}	&En liste af opsatte enheder præsenteres på skærmen 
			&Visuel: Listen vises for bruger med opsatte Enheder 
			&N/A 
			&N/A \\\hline
			 
\textbf{5.a}&Bruger indtaster adresse <1> på Enhed
			&Visuel: Muligt at skrive adresse
			&N/A 
			&N/A \\\hline
			 
\textbf{5.b}&Master deaktivere Enhed
			&Visuel: Enhed 1 deaktiveres
			&N/A 
			&N/A \\\hline
			
\textbf{5.c}&Master sletter Enhed fra systemet
			&Visuel: Enhed fjernes fra Enhedslisten 
			&N/A 
			&N/A \\\hline
			 
\textbf{6}	&Master opdaterer liste med opsatte enheder 
			&Visuel: En opdateret liste præsenteres på skærmen 
			&N/A 
			&N/A \\\hline
			 
\textbf{7}	&Bruger kan returnere til hovedmenu eller opsætte ny Enhed
			&Visuel: Det er muligt at returnere til hovedmenu eller opsætte ny Enhed
			&N/A
			&N/A \\\hline

\end{longtable}
	\label{ATUC1} 
\end{center}