%%% Tabel opsætning + Headernavn!!
\begin{center}
\begin{longtable}{|p{0,5cm}|p{3,8cm}|p{3,8cm}|p{2,2cm}|p{2,2cm}|} % l for left, c for center, r for right 
\hline
\multicolumn{5}{|l|}{\textbf{UC1: Tilføj/fjern enhed}} \\ \hline
\multicolumn{1}{|c|}{} &
\textbf{Test} &
\textbf{Forventet \newline Resultat} &
\textbf{Resultat} &
\textbf{Godkendt/ \newline Kommentar} \\ \hline 
\endfirsthead

\multicolumn{5}{l}{...fortsat fra forrige side} \\ \hline 
\multicolumn{1}{|c|}{} &
\textbf{Test} &
\textbf{Forventet \newline Resultat} &
\textbf{Resultat} &
\textbf{Godkendt/ \newline Kommentar} \\ \hline 
\endhead

%%%% Tabel Opsætning

\textbf{1}	&Bruger vælger ''Tilføj/ fjern enhed'' i hovedmenu 
			&Menu fremkommer på skærmen 
			&N/A 
			&N/A \\\hline
			 
\textbf{2}	&Bruger kan vælge "Tilføj enhed" eller "Fjern enhed" 
			&Mulighederne vises på skærmen 
			&N/A 
			&N/A \\\hline
			 
\textbf{3}	&Bruger vælger ''Tilføj enhed'' 
			&Menu for ''Tilføj enhed'' kommer frem på skærmen 
			&N/A 
			&N/A \\\hline
			 
\textbf{4}	&En liste af opsatte enheder præsenteres på skærmen 
			&Listen præsenteres på skærmen 
			&N/A
			&N/A \\\hline 
			
\textbf{4.a}&Master beder bruger om at indtaste informationer 
			&Felt fremkommer hvori information kan indtastes 
			&N/A 
			&N/A \\\hline
			
\textbf{4.b}&Bruger indtaster navn og adresse på enhed 
			&Indtastede information fremkommer på skærmen 
			&N/A 
			&N/A \\\hline
			 
\textbf{4.c}&Master tilføjer enhed til systemet 
			&Korrekt information lagers i hukommelse og bekræftelse vises på skærmen 
			&N/A 
			&N/A \\\hline
			 
\textbf{4.d}&Enhed forbindes til kommunikationsnetværket 
			&Seriel forbindelse etableres 
			&N/A
			&N/A \\\hline  
			
\textbf{4.e}&Master verificerer forbindelse til enhed 
			&Master kontrollere om enhed er korrekt opsat og giver bruger information herom 
			&N/A 
			&N/A \\\hline 
			
\textbf{5}	&Bruger vælger "Fjern enhed" 
			&Menu for "Fjern enhed" fremkommer på skærmen 
			&N/A 
			&N/A \\\hline
			 
\textbf{6}	&En liste af opsatte enheder præsenteres på skærmen 
			&Listen præsenteres på skærmen 
			&N/A 
			&N/A \\\hline
			 
\textbf{6.a}&Bruger indtaster adresse på enhed 
			&Felt til indtastning fremkommer på skærmen 
			&N/A 
			&N/A \\\hline
			 
\textbf{6.b}&Master deaktiver enheden 
			&Adressen på den indtastede enhed deaktiveres
			&N/A 
			&N/A \\\hline
			
\textbf{6.c}&Master sletter enhed fra systemet 
			&Al information om enhed slettes fra systemet 
			&N/A 
			&N/A \\\hline
			 
\textbf{7}	&Master opdatere listen men opsatte enheder 
			&Den opdateret liste præsenteres på skærmen 
			&N/A 
			&N/A \\\hline
			 
\textbf{8}	&Brugeren kan returnere til hovedmenuen eller opsætte ny enhed 
			&Master giver mulighed for at returnere eller fortsætte med at tilføje/fjerne enheder 
			&N/A
			&N/A \\\hline

\end{longtable}
	\label{ATUC1} 
\end{center}