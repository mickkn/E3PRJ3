% AT 2
\begin{center}
\begin{longtable}{|p{0,5cm}|p{3,8cm}|p{3,8cm}|p{2,2cm}|p{2,2cm}|} % l for left, c for center, r for right 
\hline
\multicolumn{5}{|l|}{\textbf{UC2: Konfig}} \\ \hline
\multicolumn{1}{|c|}{} &
\textbf{Test} &
\textbf{Forventet \newline Resultat} &
\textbf{Resultat} &
\textbf{Godkendt/ \newline Kommentar} \\ \hline 
\endfirsthead

\multicolumn{5}{l}{...fortsat fra forrige side} \\ \hline 
\multicolumn{1}{|c|}{} &
\textbf{Test} &
\textbf{Forventet \newline Resultat} &
\textbf{Resultat} &
\textbf{Godkendt/ \newline Kommentar} \\ \hline 
\endhead

\textbf{1}	&Bruger vælger ''Konfig'' i hovedmenuen
			&Liste over opsatte enheder fremkommer på skærmen
			&N/A 
			&N/A \\ \hline 
			
\textbf{2}	&Bruger vælger en enhed der ønskes konfigureret
			&Menu for konfiguration for valgte enhed vises på skærmen
			&N/A
			&N/A \\ \hline 
			
\textbf{2.a}	&Bruger vælger at afbryde konfigurationen
			&Master forlader menuen og går tilbage til hovedmenuen
			&N/A
			&N/A \\ \hline 
			
\textbf{3}	&Bruger indtaster nye værdierne for den valgte enhed
			&Indstillingsmulighederne er tilgængelige 
			&N/A
			&N/A \\ \hline 
			
\textbf{4}	&Bruger gemmer de ønskede indstillinger gemmes 
			&Indstillingerne for den valgte enhed gemmes
			&N/A
			&N/A \\ \hline 
			
\textbf{4.a}	&Bruger vælger at afbryde konfigurationen
			&Indstillingerne slettes og master går tilbage til hovedmenu. Master gemmer ikke indstillingerne
			&N/A
			&N/A \\ \hline 
			
\textbf{5}	&Der testes at den valgte enhed reagerer på de nye indstillinger
			&Enheden opererer efter de nye parameter
			&N/A
			&N/A \\ \hline 
			
\end{longtable}
	\label{ATUC2} 
\end{center}