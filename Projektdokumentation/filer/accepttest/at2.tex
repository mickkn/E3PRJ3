% AT 2
\begin{center}
\begin{longtable}{|p{0,5cm}|p{3,8cm}|p{3,8cm}|p{2,2cm}|p{2,2cm}|} % l for left, c for center, r for right 
\hline
\multicolumn{5}{|l|}{\textbf{UC2: Konfig}} \\ \hline
\multicolumn{1}{|c|}{} &
\textbf{Test} &
\textbf{Forventet \newline Resultat} &
\textbf{Resultat} &
\textbf{Godkendt/ \newline Kommentar} \\ \hline 
\endfirsthead

\multicolumn{5}{l}{...fortsat fra forrige side} \\ \hline 
\multicolumn{1}{|c|}{} &
\textbf{Test} &
\textbf{Forventet \newline Resultat} &
\textbf{Resultat} &
\textbf{Godkendt/ \newline Kommentar} \\ \hline 
\endhead

\textbf{1}	&Bruger vælger ''Konfig'' i hovedmenuen
			&GUIen går ind i Konfig-menuen 
			&N/A 
			&N/A \\ \hline 
			
\textbf{2}	&Bruger vælger den Enhed brugeren ønsker at konfigurere
			&Brugeren kan vælge en Enhed
			&N/A
			&N/A \\ \hline 
			
\textbf{3}	&Bruger indstiller de ønskede indstillinger på den valgte Enhed
			&Brugeren kan ændre indstillinger
			&N/A
			&N/A \\ \hline 
			
\textbf{4}	&Bruger vælger "Gem" 
			&Brugeren kan gemme de nye indstillinger
			&N/A
			&N/A \\ \hline 
			
\textbf{5}	&Master programmere de nye indstillinger til den valgte enhed
			&De nye indstillinger bliver programmeret på den valgte enhed og enheden udfører programmet
			&N/A
			&N/A \\ \hline 
			
\end{longtable}
	\label{ATUC2} 
\end{center}